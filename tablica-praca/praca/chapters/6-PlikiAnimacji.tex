\section{pliki z~animacjami}
Ekran obsługuje 2 formaty plików: M2F i~MXF.~Rodział ten jest poświęcony opisowi tych plików.


\subsection{Założenia projektowe dla plików M2F}
\begin{enumerate}
\item Możliwość wypisania aktualnej daty i~czasu
\item Możliwość wyświetlania grafiki
\item Możliwość wyświetlania tekstu i~grafiki jednocześnie
\item Dotatkowe możliwości, między innymi wyświetlanie w~pionie
\item Różne czcionki
\item Mały rozmiar plików
\item Mała wymagana moc obliczeniowa przy odtwarzaniu
\item Plik może nie być czytelny dla człowieka
\end{enumerate}

\subsection{Ogólny opis formatu}
W ramach realizacji założeń powstał format M2F.~Nazwa ma sugerować, że opisuje on to samo, co plik projektu MMF, tylko w~inny sposób. Cała metoda wyświetlania bazuje przede wszystkim na buforze tła i~opisie liter. Co pewien czas jest generowana nowa klatka. Na początku wczytywane są komendy sterujące, modyfikujące tło, opis liter. Potem tworzy się kopię tła i~nanosi na nią po kolei zdefiniowane litery. Litery mogą się nachodzić na siebie nadpisując się. Gotowa klatka zostaje narysowana. W~kolejnych podpunktach znajdują się opisy poszczególnych elementów formatu oraz bardziej szczegółowy opis.

\subsection{Komendy}
\subsubsection{Komendy sterujące}
Komendy sterujące jako pierwszy bajt mają wartość 0. Drug bajt jest numerem komendy. Obecnie jest dostępnych 21 komend:
\begin{itemize}
\item 0 – koniec pliku (plik po zakończeniu jest odczytywany jako ciąg zer, czyli nie musi zawierać tej komendy),
\item 1 – koniec danych dla klatki, czyli odczekaj 1 klatkę
\item 2 – odczekaj 2 klatki
\item 3 – odczekaj 5 klatek
\item 4 – odczekaj 10 klatek
\item 5 – odczekaj 20 klatek
\item 6 – odczekaj 50 klatek
\item 7 – odczekaj 100 klatek
\item 8 – czekaj do końca sekundy
\item 9 – czekaj do końca minuty
\item 10 – czekaj do końca godziny
\item 11 – czekaj do końca dnia
\item 12 – nie rysuj nic
\item 13 – rysuj tylko litery

(domyślnie animacja rozpoczyna się z~takim ustawieniem)
\item 14 – rysuj tylko tło
15 – rysuj litery i~tło
\item 16 – gdy tło nie jest rysowane, rysuj na pustej klatce

(domyślnie animacja rozpoczyna się z~takim ustawieniem)
\item 17 – gdy tło nie jest rysowane, rysuj na wypełnionej klatce
\item 18 – modyfikacja tła

(składnia jest opisana poniżej)
\item 19 – czekaj na przycisk
\item 20 – przewiń plik do początku

(ustawienia nie są przywracane do stanu początkowego)
\end{itemize}

\subsubsection{Składnia instrukcji modyfikującej tło}
Odczytywane są kolejne 2 bajty z~pliku, opisują one położenie modyfikowanego obrazu. Modyfikuje się obraz w~podziale na obszary o~wymiarach i~położeniu będących wielokrotnościami liczby 16. Pierwszy bajt podaje współrzędne początkowe, drugi współrzędne końcowe. Z~założenia bardziej znaczące bity oznaczają położenie w~pionie, jednak ze względu na wysokość wyświetlacza będącą wysokością. Wartości z~tych bajtów są mnożone przez 16. Odczytanie wartości 0x01 i~0x13 oznacza, że zmodyfikowany zostanie obszar o~współrzędnej X od 16 do 48 i~o~współrzędnej Y od 0 do 16. Obszar ten zawiera 512 pikseli, czyli jest opisany 512 bitami, czyli 64 bajtami. Każde 2 bajty stanowią szesnastobitową kolumnę obrazu, pierwszy bajt jest bardziej znaczący. Na ekranie piksele na dole odpowiadają najbardziej znaczącej wartości. Dla przykładowych bajtów 0x33 i~0xf0 postanie kolumna (od dołu) 0011001111110000. Gdyby obszar był wyższy niż 16 pikseli, dane byłyby w~pasach, na przykład najpierw X od 16 do 48, Y od 0 do 16; potem X od 16 do 48, Y od 16 do 32.

\subsubsection{Komendy modyfikujące litery}
Komendy tego rodzaju rozpoczynają się bajtem większym od zera, oznaczającym numer litery, po który jest numer parametru i~jego nowa wartość. Można modyfikować następujące parametry:
\begin{itemize}
\item 0 – kod znaku

(opis użytego kodowania znajduje się poniżej)
\item 1 – położenie w~poziomie

(może być ujemne, liczone od lewej strony)
\item 2 – położenie w~pionie

(może być ujemne, liczone od góry)
\item 3 – tryb wyświetlania

(opis trybów znajduje się poniżej)
\item 4 – czcionka

(opis czcionek poniżej 1.7)
\item 5 – przycięcie znaku w~poziomie, od ile
\item 6 – przycięcie znaku w~pionie, od ile
\item 7 – przycięcie znaku w~poziomie, do ile
\item 8 – przycięcie znaku w~pionie, do ile

(opis przycięć znajduje się poniżej)
\end{itemize}

\subsection{Kodowanie znaków}
Użyte kodowanie w~większości jest zgodne z~ASCII, to znaczy od znaku 30 (spacja) do 126 (tylda). Z~założenia to kodowanie zawiera wszystkie znaki powszechnie używane w~języku polskim. Zawiera róznież aktwne znaki słóżące do opisu aktualnej daty i~czasu. Oprócz kodowania zgodnego z~ASCII, w~ramach pozostałych kodów są umieszczone następujące znaki:
\begin{itemize}
\item 0 – liczba dziesiątek godzin w~systemie 24-godzinnym (0 – 23), spacja zamiast zera
\item 1 – liczba jedności godzin
\item 2 – liczba dziesiątek minut
\item 3 – liczba jedności minut
\item 4 – liczba dziesiątek sekund
\item 5 – liczba jedności sekund
\item 6 – liczba dziesiątek numeru dnia w~miesiącu (1 – X), spacja zamiast zera
\item 7 – liczba jedności numeru dnia w~miesiącu
\item 8 – liczba dziesiątek numeru miesiąca (1 – 12)
\item 9 – liczba jedności numeru miesiąca
\item 10 – liczba dziesiątek roku
\item 11 – liczba jedności roku
\item 12 – symbol stopnia – \( ^{\circ} \)
\item 13 – symbol euro – \euro
\item \begin{tabular}{ccccccccc}
14 – Ą & 15 – Ć & 16 – Ę & 17 – Ł & 18 – Ń & 19 – Ó & 20 – Ś & 21 – Ź & 22 – Ż \\
23 – ą & 24 – ć & 25 – ę & 26 – ł & 27 – ń & 28 – ó & 29 – ś & 30 – ź & 31 – ż \\
\end{tabular} 
\item 127 – pusty znak – \( \Box \)
\end{itemize}

\subsection{Tryby wyświetlania znaków}
Dostępne jest wiele możliwości wyświetlenia znaku. Poszczególne bity trybu opisują modyfikacje przy rysowaniu. Bity są opisane od najmniej znaczącego (pierwszy jest bitem jedności). Bit 1. i~5. są objaśnione na końcu, działają w~sposób zależny od siebie.
\begin{itemize}
\item bit 2. – litera zostanie przetransponowana, przerzucona względem głównej przekątnej
\item bit 3. – litera zostanie przerzucona w~poziomie
\item bit 4. – litera zostanie przerzucona w~pionie
\item bit 1. i~bit 5. – tło oznacza tło dla danej litery, czyli bieżącą klatkę, na której mogą być narysowane wcześniejsze litery.
\begin{itemize}
\item bit 1. wyzerowany i~bit 5. wyzerowany

Litery są nanoszone przez logiczną sumę, mogą tylko zmienić ciemne tło na jasne. Każdy piksel jest ustawiany według wyniku operacji: tło LUB litera.
\item bit 1. ustawiony i~bit 5. wyzerowany

Litery są nanoszone przez logiczny iloczyn z~negatywem litery. Można to określić jako odjęcie od dotychczasowego tła. Litera może tylko zmienić jasne tło na ciemne, każdy piksel jest ustawiany według wyniku operacji: tło I~(NIE litera).
\item bit 1. wyzerowany i~bit 5. ustawiony

Wartości pikseli z~litery są nanoszone niezależnie od tła. Ustawienie znaku spacji oznacza że znak w~ogóle nie jest rysowany, co jest nadrzędne względem ustawień trybu.
\item bit 1. wyzerowany i~bit 5. wyzerowany

Litery są nanoszone jako alternatywa wykluczająca z~tłem. Tam gdzie litera jest jasna, wartość tła jest negowana.
\end{itemize}
\end{itemize}

1.11
\subsection{Użyte czcionki}
Zastosowano 3 czcionki, o~identyfikatorach 0, 1 i~2. Wszystkie litery czcionek są wpisane w~kwadrat o~danym rozmiarze. Oznacza to, że mają stałą szerokość, ale nie oznacza, ze nie można ustawić znaków bliżej i~zależnie od kształtu litery, uzyskując czcionkę proporcjonalną.

Czcionka 0 ma rozmiar 16px. W~tym projekcie oznacza to, że każda litera mieści się w~kwadracie 16 na 16 pikseli. Głównym założeniem przy tworzeniu tej czcionki, było żeby uzyskać możliwie dużą czytelność. Zostało to uzyskane przez stworzenie wielkich liter i~cyfr które mają wysokość 16 pikseli. Z~założenia gdy litery tą czcionką są wyświetlane, stanowią jedyną linię. Nie jest więc problemem, że 2 line takiego tekstu jedna nad drugą by się stykały . Powstałym problemem są więc różnego rodzaju „kropki”, „kreski” i~„ogonki” nie mieszczące się normalnie między linią podstawową fonta, a~linią górną. Problematyczne znaki zostały więc „ściśnięte”, co powoduje, że mogą wyglądać gorzej i~być mniej czytelne. 

Czcionka 1 ma rozmiar 8px. Celem utworzenia tej czcionki była możliwość wyświetlenia możliwie czytelnych dwóch linii tekstu. Wielkie litery i~cyfry mają jeden piksel odstępu od góry. Znaki z~elementami wystającymi w~górę korzystają z~tej przestrzeni, jak również stanowi ona odstęp między liniami. Tak jak w~czcionce 0, linia dolna i~linia podstawowa są sobie równe.

Czcionka 2 ma rozmiar 12px. Bazuje na czcionce Fixedsys zaprojektowanej przez Microsoft®. Celem zastosowania tej czcionki jest wypełnienie luki w~rozmiarach między poprzednimi czcionkami. Można przypuszczać, że przy opracowywaniu tej czcionki wygląd był ważnym czynnikiem, a~jednocześnie jest ona jednocześnie bardzo czytelna. Niektóre znaki w~tej czcionce były za wysokie i~zostały obniżone. Przez to również w~niej linia podstawowa nie jest zachowana we wszystkich znakach.

\subsection{Zasady przycięć znaków}
Przycięcia OD i~DO w~pionie jak i~w~poziomie odnoszą się do pikseli adresowanych od 0 do 16 (dla czcionki o~wysokości 16) i~są liczone względem współrzędnych litery. Gdy nie mają przycinać znaku, mają wartość OD = 0 i~DO = 15. Zwiększenie wartości OD powoduje przycięcie od góry, zmniejszenie wartości DO, przycięcie od dołu. W~związku z~tym, gdy wartości OD i~DO są równe, zostaną pozostawione tylko piksele o~tym adresie. Gdy wartość OD jest większa od wartości DO, z~czcionki pozostaną 2 pasy, od 0 do OD i~od DO do 15. Tym samym wartości  OD = 15 i~DO = 0 oznaczają, że pozostaną piksele adresie 0 i~o~adresie 15.

\subsection{Szczegółowy opis działania}
\subsubsection{Stan początkowy}
\begin{enumerate}
\item tło jest wyczyszczone – ustawiane na ciemne
\item rysowanie liter jest włączone
\item rysowanie tła jest wyłączone
\item gdy nie ma tła, zaczyna się od ciemnej klatki
\item ustawienie początkowe dla liter:
\begin{enumerate}
\item wszystkie litery są spacjami (kod 32)
\item przycięcia są ustawione na OD = 0 i~DO = maksymalna wartość
\item tryb wyświetlania ustawiony na 0, czyli suma logiczna z~tłem, bez transpozycji o~przerzuceń
\item czcionka ustawiona na czcionkę 0, 16 pikseli
\end{enumerate}
\end{enumerate}
algorytm:
\begin{enumerate}
\item Czytaj komendy aż do natrafienia na komendę wymuszającą czekanie.
\item Zależnie od ustawień, kopiuj tło do bieżącej klatki, lub ustaw ją na pełną bądź pustą.
\item Po kolei, dla każdej litery o~numerze od 1 do 127 (litery mają numery od  wykonuj zależnie od ustawień (kolejność ma znaczenie):
\begin{enumerate}
\item Nadrzędną metodą ukrycia znaku jest ustawienie jego kodu na 32, czyli spację. Nawet w~trybie nadpisywania tła, taki znak zostanie pominięty.
\item Transpozycja.
\item Przerzucenie w~poziomie o~pionie.
\item Przycięcia w~poziomie i~pionie.
\item Narysuj literę modyfikując bieżącą klatkę
\end{enumerate}
\item Przekaż bieżącą klatkę funkcji rysującej
\item Czekaj do końca klatki, do końca 25~ms które powinna trwać.
\item Jeżeli nie nastąpiło oczekiwane zdarzenie (np. przycisk, zmiana godziny, narysowanie 100 klatek) przejdź do punktu 2.
\item Jeżeli nastąpiło oczekiwane zdarzenie przejdź do punktu 1.
\end{enumerate}

\subsection{Przykłady plików}
tudu

\subsection{Podsumowanie i~możliwości rozwoju}
Pliki w~formacie M2F pozwalają wyświetlić bardzo różnorodne dane. Format może zostać rozszerzony o~kolejne właściwości znaków i~komendy, również takie, jak kolor znaku, obsługa większych obszarów, modyfikacja grup znaków, skoki warunkowe i~wiele innych. Do kodowania znaków również można dodać kolejne dane, na przykład dzień tygodnia i~temperaturę. Pliki M2F są jednak dość skomplikowane. Trudno jest uzyskać w~tym formacie nawet nieskomplikowane animacje w~sposób zautomatyzowany. Zaawansowane wykorzystanie jest bardzo skomplikowane. Z~kolei ręczne tworzenie tych plików, w~edytorze szesnastkowym, jest bardzo czasochłonne i~niepraktyczne. 
\subsection{Pliki MXF}
\subsubsection{Założenia}
W pliku M2F tak podstawowy efekt jak przesuwający się w~poziomie napis wymaga wielu operacji, czyli w~każdej klatce zmian położenia każdej litery, oraz przekładania wychodzących liter na początek. Przesunięcie tekstu o~jeden piksel wymaga około 50 bajtów. W~związku z~tym został określony drugi format, czyli MXF.~Jego nazwa ma nawiązywać do formatu tekstowego TXT. W~przeciwieństwie do poprzednio opisanego formatu, ten format jest bardzo prosty.

\subsubsection{Opis}
W tym formacie można wyświetlać tylko i~wyłącznie przesuwający się w~lewo tekst. Kolejne znaki są czytane z~pliku i~wsuwane od prawej strony. Wykorzystana jest czcionka 0, o~wysokości 16 pikseli. Kodowanie znaków jest takie samo jak w~plikach M2F (opisane w~punkcie 1.5), czyli plik może zawierać aktualną godzinę i~datę. Ze względu na sposób implementacji, w~którym narysowane znaki nie są później modyfikowane, nie ma większego sensu umieszczanie liczby jedności sekund. Zamiast tego znaku (o~kodzie 5) jest synchronizacja zegara z~zegarem czasu rzeczywistego. Gdyby zegar aktualizował się samoczynnie wyświetlona godzina mogłaby być niespójna, zmienić się pomiędzy wstawieniem poszczególnych liter.
