
\chapter{Zakończenie}

W ramach niniejszej pracy zaprojektowano oraz zaimplementowano system służący do wymiany zarówno plików, jak i komentarzy. System \emph{TeamSync} w części transportu danych opiera się na aplikacji \emph{BitTorrent Sync} działającej na protokole BitTorrent. Komunikację pomiędzy systemami umożliwia API stworzone przez producentów aplikacji \emph{BitTorrent Sync}, które umożliwia kontrolowanie synchronizowanych danych. Wewnątrz stworzonego systemu \emph{TeamSync} zostało zaimplementowanych wiele funkcjonalności zarówno podstawowych (wymiana komentarzy), jak i dodatkowych (wyszukiwanie, filtrowanie). Wszystkie najistotniejsze funkcje systemu \emph{TeamSync} zostały wymienione poniżej.

Użytkownicy mogą zarządzać (dodawać, modyfikować oraz usuwać) folderami, które współdzielą z innymi użytkownikami w sieci. System \emph{TeamSync} zapewnia pełną nawigację po synchronizowanych katalogach z poziomu interfejsu użytkownika. Interfejs dodatkowo umożliwia tworzenie przez użytkowników nowych wątków, wypowiadanie się wewnątrz nich oraz edycję komentarzy wraz z możliwością przeglądania historii zmian. Tworzone wątki mogą (lecz nie muszą) dotyczyć dowolnych plików zawartych w folderze.

Poza operacjami dokonywanymi na komentarzach system umożliwia swobodne wyszukiwanie informacji, poprzez narzędzia sortujące i filtrujące wypowiedzi oraz wątki. Dyskusje mogą zostać posortowane alfabetycznie, chronologicznie, według największej ilości postów oraz według najpóźniej dodanej odpowiedzi. Użytkownik ma możliwość filtrować wątki według procentowego udziału wybranych użytkowników w dyskusji. System również umożliwia wyszukiwanie wpisywanych przez użytkownika fraz w komentarzach z możliwością zmiany zbioru komentarzy --- wątek, wszystkie komentarze, wszystkie komentarze konkretnego użytkownika. Dostępny jest też tryb statystyk, w którym wyświetlana jest ilość wypowiedzi każdego z użytkowników oraz ich procentowy udział w zbiorze komentarzy.

Szczegółowy model systemu, jego architektura, ogólny sposób działania, konfiguracja oraz impementacja zostały opisane w poprzednich rozdziałach niniejszej pracy. 

System \emph{TeamSync} został zaimplementowany w sposób umożliwiający jego dalszy rozwój. Wśród dodatkowych funkcjonalności, które mogłyby zostać zaimplementowane bez konieczności dokonywania dużych zmian w strukturze kodu (zarówno części klienckiej, jak i serwerowej) są:

\begin{enumerate}[noitemsep]
  \item Wykrywanie typów plików obrazów lub muzycznych (np. \texttt{jpg}, \texttt{mp3}) i możliwość ich podglądu (lub odtworzenia w przypadku plików muzycznych).
  
  \item Dostosowanie aplikacji TeamSync do pracy z systemami operacyjnymi: Windows oraz MacOS.
  
  \item Możliwość dokonywania zmian w synchronizowanym folderze (np. wyłączenie tzw. ,,trackera'', wyszukiwanie innych użytkowników za pomocą DHT \cite{dht}.
  
  \item Dodanie kilku dodatkowych ustawień dotyczących graficznego interfejsu:
  \begin{itemize}[noitemsep]
    \item możliwość zmiany formatu daty wyświetlanej obok komentarzy,
    \item możliwość definiowania własnego koloru czcionki lub ramki, wewnątrz której wyświetlane są komentarze użytkownika,
    \item możliwość wybrania przez użytkownika pliku graficznego (tzw. ,,avatara'') umieszczanego obok jego nazwy na liście użytkowników oraz obok każdego komentarza.
  \end{itemize}
\end{enumerate}