
% Szkielet dla pracy pisanej w języku polskim.

\documentclass[polish,a4paper,twoside]{ppfcmthesis}


\usepackage[utf8]{inputenc}
\usepackage[OT4]{fontenc}
\usepackage{enumitem}

\newcommand{\todo}[1]{}
\renewcommand{\todo}[1]{{\color{red} TODO: {#1}}}
\def\labelitemi{--}

\authortitle{}                                        % You can place "inż.~" here, if you really want to.
\author{Filip Rachwalak}                              % Your name comes here
\title{TeamSync --- System wymiany danych i komentarzy w systemach P2P}                   % Note how we protect the final title phrase from breakings
\ppsupervisor{dr inż. Anna Kobusińska}                % Your supervisor comes here.
\ppyear{2015}                                         % Year of final submission (not graduation!)

\addto\captionspolish{\renewcommand{\figurename}{Rys.}}

\brokenpenalty10000\relax

\begin{document}

% Front matter starts here
\frontmatter\pagestyle{empty}%
\maketitle\cleardoublepage%

% Blank info page for "karta dyplomowa"
\thispagestyle{empty}\vspace*{\fill}%
\begin{center}Tutaj przychodzi karta pracy dyplomowej;\\oryginał wstawiamy do wersji dla archiwum PP, w pozostałych kopiach wstawiamy ksero.\end{center}%
\vfill\cleardoublepage%

% Table of contents.
\pagenumbering{Roman}\pagestyle{ppfcmthesis}%
\tableofcontents* \cleardoublepage%

% Main content of your thesis starts here.
\mainmatter%
\chapter{Wstęp}

W dobie coraz łatwiejszego i szybszego dostępu do sieci globalnej oraz rosnącej liczby urządzeń przetwarzających dane zachodzi coraz pilniejsza potrzeba sprawnej i bezpiecznej wymiany informacji. Twórcy obecnych narzędzi oferują coraz więcej dodatkowych funkcjonalności towarzyszących wymianie danych. Mogą się one różnić w zależności od architektury stworzonego systemu, ze względu na ograniczenia, które architektura ze sobą niesie. Głównym aspektem rozpatrywanym w ramach niniejszej pracy jest wymiana danych w postaci plików różnego typu, udostępnianych wzajemnie pomiędzy użytkownikami systemu.

Jedną z ważnych dodatkowych funkcji systemów umożliwiających wymianę danych jest możliwość komentowania zamieszczanych plików przez użytkowników, którzy udostępniają swoje dane. W przypadku systemów scentralizowanych jest to bardzo często dodawana usługa, ponieważ implementacja oraz utrzymanie takich systemów są zdecydowanie łatwiejsze. W centralnym systemie komunikaty oraz pliki można przechowywać i odczytywać w łatwy sposób w centralnej części systemu. Jednak dane umieszczane w takich systemach są mniej bezpieczne i narażone na brak dostępu w przypadku awarii części centralnej.

Chcąc wymieniać (i przechowywać) dane bezpieczniej, można wybrać rozwiązania o rozproszonej architekturze. Aplikacja \emph{TeamSync} zaimplementowana w ramach niniejszej pracy magisterskiej jest narzędziem, które łączy funkcjonalność synchronizacji danych z możliwością wymiany opinii na ich temat poprzez system komentarzy pogrupowanych w sposób ułatwiający ich umieszczanie oraz czytanie. W swojej ostatecznej postaci umożliwia wygodne wprowadzanie swoich opinii przez użytkowników oraz łatwe i szybkie wyszukiwanie informacji wewnątrz nich. Poprzez wygodny interfejs graficzny umożliwiona została nawigacja po synchronizowanych katalogach i możliwość wybrania pliku, o którym można rozpocząć dyskusję. Funkcjonalność systemu obejmuje również bardziej zaawansowane operacje wykonywane na danych --- filtrowanie oraz sortowanie wątków i funkcje wyszukiwania wpisywanych fraz wewnątrz komentarzy.

System \emph{TeamSync} do komunikacji oraz wymiany danych wykorzystuje narzędzie \emph{BitTorrent Sync}. Po uruchomieniu aplikacji \emph{TeamSync} w tle rozpoczyna działanie system \emph{BitTorrent Sync} i obsługuje wymianę danych pomiędzy użytkownikami.

Struktura pracy jest następująca. W rozdziale 2 przedstawiono zestawienie oraz krótki opis kilku istniejących rozwiązań, które swoją funkcjonalnością zbliżone są do aplikacji zaimplementowanej w ramach niniejszej pracy. Wśród nich znajduje się wspomniany powyżej \emph{BitTorrent Sync}. Rozdział 3 zawiera opis modelu systemu --- przyjęte założenia, nałożone ograniczenia oraz podstawowe obiekty i struktury opisywane w pracy --- komentarze, wątki, foldery. Omawiana jest również wewnątrz niego spójność synchronizowanych danych.

Rozdział 4 poświęcony jest architekturze systemu. Opisana zostały ogólna idea działania, protokół komunikacyjny z aplikacją \emph{BitTorrent Sync} oraz dokładne dane, które są przechowywane w plikach komentarzy oraz wątków. W rozdziale 5 omówiono działanie systemu --- jak wyglądają tworzone struktury folderów w systemie plików oraz pliki konfiguracyjne przechowujące dane aplikacji.

Rozdział 6 zawiera opis zastosowanych w projekcie technologii oraz sposób, w jaki część kliencka i serwerowa komunikują się pomiędzy sobą. Dodatkowo w tym rozdziale wyjaśnione zostały przykładowe polecenia wykonywane przez serwer oraz sposób działania użytych technologii. Rozdział 7 stanowi podsumowanie pracy wraz z wnioskami oraz możliwymi kierunkami rozwoju aplikacji.

W dodatku A znajduje się opis graficznego interfejsu użytkownika wraz ze szczegółami dotyczącymi trybu jego pracy oraz wskazówki dotyczące stosowania filtrów wewnątrz aplikacji. W dodatku znajdują się również przykłady użycia wybranych funkcji interfejsu graficznego.
\input{chapters/02-przeglad-rozwiazan.tex}
\input{chapters/03-model-systemu.tex}
\chapter{Ogólna koncepcja i architektura}

Aplikacja \emph{TeamSync} zrealizowana w ramach niniejszej pracy ma strukturę dwuwarstwową. Strony serwera oraz klienta, na wzór serwisów internetowych, są od siebie odizolowane, a komunikacja między nimi zachodzi z wykorzystaniem protokołu \texttt{HTTP} \cite{http} \cite{httparticle}. BitTorrent Sync jest narzędziem współpracującym z aplikacją, które jest odpowiedzialne za wymianę danych (synchronizację). Komunikacja z tym narzędziem jest możliwa dzięki odpowiednio skonfigurowanemu API, które zostało udostępnione przez twórców całego systemu.

Działanie aplikacji \emph{TeamSync} polega na uruchomieniu w tle systemu synchronizującego dane i udostępnienie użytkownikowi wygodnego interfejsu do umieszczania i odczytywania komentarzy, co odbywa się poprzez odpowiednią manipulację plikami wewnątrz synchronizowanych folderów. Jedyną drogą wymiany danych jest wspomniany system BitTorrent Sync.

\begin{figure}[htb]
  \vspace{5pt}
  \begin{center}
    \includegraphics[width=380pt]{figures/architecture3.png}
  \end{center}
  \caption{Uproszczona architektura systemu \emph{TeamSync}.}
  \label{architecturepic}
\end{figure}

Poniżej zostaną omówione w oddzielnych podrozdziałach poszczególne moduły architektury systemu: BitTorrent Sync, system \emph{TeamSync} wraz z komunikacją pomiędzy jego częścią serwerową a kliencką oraz serwer NTP.

\section{BitTorrent Sync}

W poniższej sekcji opisanych zostanie kilka aspektów zarówno systemu BitTorrent Sync, jak i zaimplementowanemu przez jego twórców API, dzięki któremu możliwe jest wykonywanie większości funcji tego systemu z poziomu kodu. Rozdział rozpocznie sekcja \emph{secret} opisująca łańcuch znaków wymieniany między użytkownikami, po której zostanie przedstawiony bardziej szczegółowy opis protokołu komunikacji z systemem BitTorrent Sync.

\subsection{Secret}

\label{secret}

W aplikacji BitTorrent Sync \emph{secret} jest ciągiem $32$ znaków, służącym do identyfikacji synchronizowanych folderów. Dodając katalog do tego systemu, użytkownik może posłużyć się istniejącym już \emph{secretem}, jeśli otrzymał go od użytkownika, który wcześniej zainicjował folder. Może też sam go zainicjować, generując \emph{secret} tworząc losowy ciąg znaków odpowiadający wyrażeniu regularnemu \texttt{[A-Z0-9]\{32\}}, co odpowiada trzydziestu dwóm losowym znakom wybranym ze zbioru cyfr od $0$ do $9$ oraz liter alfabetu od \texttt{A} do \texttt{Z}.

Duża ilość znaków ($32$ znaki) oraz duży ich zbiór ($24$ litery $+$ $10$ cyfr daje łącznie moc zbioru znaków równą $34$) powodują, że istnieje znikoma szansa na powtórzenie się dwóch łańcuchów znakowych.

Podczas generowania wartości \emph{secret}, tworzone są dwa łańcuchy znaków o dwóch typach synchronizacji:

\begin{description}[noitemsep]
  \item[read\_write] --- użytkownicy, którzy otrzymają \emph{secret} pochodzący z tej wartości będą mogli zarówno odczytywać dane, jak i je modyfikować oraz dodawać swoje; modyfikacje danych --- w przeciwieństwie do trybu \texttt{read\_only} --- będą się propagować do pozostałych użytkowników,
  
  \item[read\_only] --- użytkownicy, którzy otrzymają ten \emph{secret}, będą mogli wyłącznie pobierać dane, nie mogąc ich modyfikować (niezależnie od modyfikacji, zmiany nie będą propagowane do innych użytkowników).
\end{description}

W zależności od tego, który z \emph{secretów} użytkownik otrzyma, tak system będzie uspójniał dane.

\subsection{Protokół BitTorrent Sync API}

\label{btsyncapiproto}

Twórcy systemu BitTorrent Sync umożliwili korzystanie z dodatkowej funkcjonalności aplikacji nie tylko użytkownikom, ale również programistom chcącym testować lub udoskonalać narzędzia związane z wymianą plików. Sterowanie systemem z poziomu kodu jest możliwe dzięki zaimplementowanemu przez twórców aplikacji API, które umożliwia komunikację pomiędzy narzędziem a stworzoną przez programistę aplikacją.

API narzędzia odbiera żądania --- z odpowiednio dobranymi parametrami --- nasłuchując na porcie wskazanym przez użytkownika w pliku konfiguracyjnym (dokładny opis pliku konfiguracyjnego znajduje się w podrozdziale \ref{configbtsync}). Pełen adres, na który programista musi skierować żądanie HTTP ma postać wzorca:
\begin{verbatim}
                          http://<adres>:<port>/api
\end{verbatim}

W przypadku aplikacji \emph{TeamSync} zaimplementowanej w ramach niniejszej pracy magisterskiej, ze względu na lokalizację systemu BitTorrent Sync (lokalna maszyna) użyty adres to \texttt{localhost} oraz port \texttt{8787} (zmieniony z domyślnego \texttt{8888} ze względu na jego dużą popularność w aplikacjach). Pełen adres, na który są wysyłane żądania pomiędzy aplikacją \emph{TeamSync}, a API systemu BitTorrent Sync to:
\begin{verbatim}
                          http://localhost:8787/api
\end{verbatim}

\subsubsection*{Komunikaty BitTorrent Sync API}

Żądania wysyłane z systemu \emph{TeamSync} do API zawierają w nagłówku parametr uwierzytelniający \emph{Basic HTTP Authentication} \cite{basicauth}. Dane dostępu zawierają wartości \texttt{login} oraz \texttt{password}, które --- aby zostały poprawnie odebrane --- muszą odpowiadać danym umieszczonym wewnątrz pliku konfiguracyjnego systemu BitTorrent Sync. Uruchamiając po raz pierwszy aplikację \emph{TeamSync} dane uwierzytelniające są tworzone wraz z plikami konfiguracyjnymi i ustawione domyślnie na \texttt{team} oraz \texttt{sync} odpowiednio dla wartości \texttt{login} oraz \texttt{password}.

Zapytania przyjmowane przez BitTorrent Sync API zawierają dodatkowo tzw. ,,metodę'' wraz z jej argumentami. Metoda oznacza funkcję, która zostanie wykonana na podanych argumentach np. pobieranie listy folderów wprowadzonych do systemu i współdzielonych z innymi użytkownikami, pobieranie listy użytkowników, z którymi użytkownik synchronizuje dany folder, zwrócenie ustawień konkretnego folderu lub ich zmiana i wiele innych. W aplikacji \emph{TeamSync} użyto niewielkiej ilości możliwych metod ze względu na ich dużą liczebność i użyteczność odbiegającą od głównego celu projektu.

Użyte metody wraz z typem przyjmowanych argumentów zostaną opisane w oddzielnych sekcjach. Wszystkie zwroty użyte w poniższych podrozdziałach dotyczące aplikacji takie jak: ,,aplikacja'', ,,system'', ,,narzędzie'', będą dotyczyły systemu BitTorrent Sync. Sposób działania narzędzia zaimplementowanego w ramach niniejszej pracy nie będzie omawiany podczas opisu poniższych metod.

Przykłady dotyczące sposobu wysyłania zapytań przez aplikację \emph{TeamSync} są napisane według wzoru podyktowanego przez wymagania funkcji pakietu \texttt{requests} języka \emph{Python}, które zostały użyte do wysyłania żądań do BitTorrent Sync API.

\begin{minipage}{\linewidth}
\vspace{15pt}
\begin{verbatim}
            requests.get('http://adres:port/api',
                         auth=('login', 'password'),
                         params={
                             'method': 'nazwa_metody',
                             'arg1': 'wartość_argumentu',
                             [ . . . ]
                         }
            )
\end{verbatim}
\vspace{15pt}
\end{minipage}

Pierwszy argument wewnątrz zapytania reprezentuje adres wraz z numerem portu, na który zostanie wysłane żądanie. Dane uwierzytelniające przesyłane są wewnątrz parametru \texttt{auth}, natomiast nazwa metody oraz pozostałe argumenty przesyłane są wewnątrz parametru \texttt{params}, który ma strukturę słownika z nazwą oraz wartością poszczególnych argumentów.

\subsubsection*{\emph{get\_secrets}}

\label{getsecrets}

Metoda \emph{get\_secrets} nie wymaga żadnych dodatkowych argumentów i służy do generowania przez aplikację ciągów znakowych identyfikujących synchronizowany folder.  Zapytanie wysyłane do BitTorrent Sync API w celu ich uzyskania:

\begin{minipage}{\linewidth}
\vspace{15pt}
\begin{verbatim}
               requests.get('http://localhost:8787/api',
                            auth=('team', 'sync'),
                            params={
                                'method': 'get_secrets',
                            }
               )
\end{verbatim}
\vspace{15pt}
\end{minipage}

Odpowiedź aplikacji będąca słownikiem w formacie \texttt{JSON} \cite{jsonarticle} zawiera dwie wartości:

\begin{itemize}[noitemsep]
  \item \emph{read\_write} --- \emph{secret} służący do synchronizacji folderu w trybie zapisu/odczytu,
  \item \emph{read\_only} --- \emph{secret}, za pomocą którego użytkownik może udostępnić swoje dane, nie pozwalając na modyfikację lub dodawanie danych innym użytkownikom.
\end{itemize}

\subsubsection*{\emph{add\_folder}}

Metoda dodająca wskazany w argumencie folder do systemu. Po otrzymaniu żądania zawierającego tę metodę, do wybranego katalogu --- który w momencie wykonywania metody musi być pusty (wymaganie systemu BitTorrent Sync) --- aplikacja dodaje ukryty folder \texttt{.sync} (dokładny opis zawartości katalogu \texttt{.sync} znajduje się w podrozdziale \ref{directorystructure}) i od tego momentu katalog wskazany w żądaniu będzie włączony do synchronizacji.

Oprócz metody komunikat zawiera dodatkowe parametry:

\begin{itemize}[noitemsep]
  \item \emph{dir} --- pełna ścieżka do folderu,
  \item \emph{secret} --- losowy ciąg znaków będący tzw. ,,secretem'', za pomocą którego możliwa jest synchronizacja katalogów między użytkownikami. Jeśli parametr ten pozostanie pusty, system otrzyma informację, że folder jest nowy i nie jest jeszcze synchronizowany przez żadnego z użytkowników. Aplikacja wygeneruje secret i od tego momentu użytkownik może przekazywać go innym węzłom, aby współdzielić wskazany katalog.
\end{itemize}

Zakładając, że synchronizowany ma zostać katalog o ścieżce \texttt{/home/user/testowy\_katalog}, zapytanie zostanie wysłane z serwera aplikacji \emph{TeamSync} w następujący sposób:

\begin{minipage}{\linewidth}
\vspace{15pt}
\begin{verbatim}
     requests.get('http://localhost:8787/api',
                  auth=('team', 'sync'),
                  params={
                      'method': 'add_folder',
                      'dir': '/home/user/testowy_katalog',
                      'secret': 'A3LL43HJ257YCKMOLAD7QSEAS7U373BVO'
                  }
     )
\end{verbatim}
\vspace{15pt}
\end{minipage}

W przypadku otrzymania takiego komunikatu z pustym argumentem \texttt{secret}, aplikacja wygenerowałaby nowy secret i zwróciłaby go w odpowiedzi. W innym przypadku, w odpowiedzi przesyłany jest wyłącznie kod błędu wewnątrz zmiennej \texttt{error}. W przypadku otrzymanej wartości równej $0$ użytkownik jest informowany, że operacja dodawania nowego folderu przebiegła pomyślnie.

\subsubsection*{\emph{get\_folders}}

Za pomocą tej metody otrzymywana jest lista folderów, które zostały wprowadzone do systemu. Podobnie jak przy metodzie \emph{get\_secrets} żadne dodatkowe argumenty nie są wymagane.

\begin{minipage}{\linewidth}
\vspace{15pt}
\begin{verbatim}
               requests.get('http://localhost:8787/api',
                            auth=('team', 'sync'),
                            params={
                                'method': 'get_folders',
                            }
               )
\end{verbatim}
\vspace{15pt}
\end{minipage}

W odpowiedzi na zapytanie \emph{get\_folders} BitTorrent Sync API zwraca listę, wewnątrz której znajdują się informacje o wszystkich synchronizowanych w systemie folderach w formacie \texttt{JSON}. Wśród kluczy słowników są:

\begin{itemize}[noitemsep]
  \item \emph{error} --- w przypadku niepowodzenia przesyłana jest tylko ta wartość ustawiona na $1$; w przypadku powodzenia wartość $0$,
  
  \item \emph{dir} --- bezwzględna ścieżka folderu w systemie plików lokalnego użytkownika,
  
  \item \emph{files} --- liczba plików, które znajdują się wewnątrz folderu,
  
  \item \emph{size} --- łączny rozmiar plików wewnątrz folderu w bajtach,
  
  \item \emph{secret} --- łańcuch znaków służący do identyfikacji współdzielonych folderów między zdalnymi użytkownikami (więcej w sekcji \ref{secret}),
  
  \item \emph{type} --- rodzaj synchronizacji; możliwe dwie wartości:
  \begin{itemize}[noitemsep]
    \item \emph{read\_write} --- użytkownicy mogą zarówno odczytywać, jak i modyfikować pliki,
    
    \item \emph{read\_only} --- tylko właściciel może modyfikować, pozostali użytkownicy wyłącznie odczytywać,
  \end{itemize}
  
  \item \emph{down\_speed} --- chwilowa prędkość pobierania; jeśli wszystkie dane w folderze są zsynchronizowane, wartość $0$,
  
  \item \emph{up\_speed} --- chwilowa prędkość wysyłania; jeśli wszystkie dane w folderze są zsynchronizowane, wartość $0$,
  
  \item \emph{id} --- unikalny identyfikator folderu, który może różnić się pomiędzy użytkownikami.
\end{itemize}

Przykładowa odpowiedź BitTorrent Sync API na powyższe żądanie może wyglądać następująco:

\begin{minipage}{\linewidth}
\vspace{15pt}
\begin{verbatim}
         [
             {
                 "error": 0, 
                 "dir": "/home/user/testowy katalog", 
                 "files": 9, 
                 "size": 3079, 
                 "secret": "A3LL43HJ257YCKMOLAD7QSEAS7U373BVO", 
                 "type": "read_write", 
                 "down_speed": 0, 
                 "up_speed": 0, 
                 "id": 902958156496830188
             }
         ]
\end{verbatim}
\vspace{15pt}
\end{minipage}

Otrzymanie takiej odpowiedzi wskazuje na fakt, iż jednym (i jedynym, ze względu na obecność tylko jednego obiektu typu \texttt{JSON} w liście katalogów) z synchronizowanych folderów w systemie jest folder o ścieżce \texttt{/home/user\-/testowy katalog}. Został on poprawnie odczytany przez system BitTorrent Sync (\texttt{'error': 0}) i zawiera dziewięć plików o łącznym rozmiarze $3079$ bajtów. Aby uwspólnić go z innymi użytkownikami w trybie umożliwiającym im zarówno odczyt, jak i modyfikację zawartości (tryb \texttt{read\_write}), należy przekazać \emph{secret} o wartości \texttt{A3LL43HJ\-257YCKMO\-LAD7QSEAS\-7U373BVO}.

Wartości \texttt{down\_speed} oraz \texttt{up\_speed} sugerują brak wymiany danych między użytkownikami w chwili wysyłania żądania \emph{get\_folders}. Wewnętrznym identyfikatorem folderu w systemie lokalnym użytkownika jest liczba \texttt{902958156496830188}.

\subsubsection*{remove\_folder}

Ostatnią z używanych przez system \emph{TeamSync} metod jest metoda \emph{remove\_folder}, która usuwa z systemu wskazany synchronizowany folder --- nie usuwa fizycznie katalogu z systemu plików, lecz blokuje jego dalszą synchronizację. Aby ponownie użyć tego samego katalogu, należy usunąć z niego wszystkie pliki (wymaganie aplikacji dotyczące pustego folderu). W parametrze żądania --- oprócz metody --- aplikacja potrzebuje wartość \emph{secret} folderu, który użytkownik zamierza usunąć.

\begin{minipage}{\linewidth}
\vspace{15pt}
\begin{verbatim}
       requests.get('http://localhost:8787/api',
                    auth=('team', 'sync'),
                    params={
                        'method': 'remove_folder',
                        'secret': 'A3LL43HJ257YCKMOLAD7QSEAS7U373BVO'
                    }
       )
\end{verbatim}
\vspace{15pt}
\end{minipage}

W przeciwieństwie do metody \emph{add\_folder}, w której parametr \emph{secret} był opcjonalny, tutaj konieczne jest przesłanie tego parametru. Jeśli nie zostanie on przesłany lub okaże się niepoprawny, aplikacja zwróci kod błędu w zmiennej \texttt{error}. W przypadku poprawnego przebiegu operacji \emph{remove\_folder}, kod błędu będzie równy $0$.

\section{\emph{TeamSync}}

\label{teamsyncarch}

Po opisaniu modułu BitTorrent Sync API z Rys. \ref{architecturepic} umieszczonego no początku rozdziału, przedstawiony zostanie dokładniej sposób działania systemu \emph{TeamSync} --- jak zapisywane są komentarze oraz wątki, jak są przekazywane w sieci oraz w jaki sposób system unika konfliktów.

\subsection{Komentarze}

\label{archcomments}

Rozproszona architektura nie daje pewności, że modyfikacja pliku odbędzie się bezkonfliktowo, ponieważ zawsze może wystąpić sytuacja, w której inny węzeł ,,równocześnie'' (pod względem zegara logicznego, nie czasu rzeczywistego) nie dokona edycji danych. Podczas testów aplikacji BitTorrent Sync, która jest odpowiedzialna za wymianę danych --- a więc również za rozwiązywanie konfliktów --- nie wyizolowano priorytetów określających, którą wersję i z którego węzła należy usunąć, a z którego rozpropagować.
 
Z powyżej opisanych powodów umieszczanie komentarzy w aplikacji \emph{TeamSync} polega na \emph{dodawaniu} nowych plików przez użytkowników. Algorytm dodawania nowego komentarza z lokalnej perspektywy wygląda następująco:

\begin{enumerate}[noitemsep]
 \item Pobierz znacznik czasowy z serwera NTP.
 
 \item Umieść pobrane od użytkownika dane (treść komentarza, identyfikator) oraz pobrany wcześniej znacznik czasowy do zmiennej o strukturze słownika (JSON).
 
 \item Zapisz słownik do pliku o nazwie złożonej ze znacznika czasowego oraz identyfikatora użytkownika wewnątrz katalogu z wątkiem.
\end{enumerate}

Umieszczenie komentarza w systemie i rozpropagowanie go pomiędzy wszystkich aktywnych użytkowników zaprezentowane jest na rysunku \ref{rys:writecomment}.

\begin{figure}[t]
  \vspace{5pt}
  \begin{center}
    \includegraphics[width=220pt]{figures/writecomment.png}
  \end{center}
  \caption{Uproszczony schemat dodawania nowego komentarza do systemu i odczytanie go przez aktywnych użytkowników.}
  \label{rys:writecomment}
\end{figure}

\subsubsection*{Struktura komentarzy}

Wszystkie zapisywane komentarze mają swoją strukturę, która --- podobnie jak w przypadku wszystkich danych przesyłanych i zapisywanych do plików w aplikacji \emph{TeamSync} --- jest strukturą słownikową zapisaną w formacie JSON. Każdy plik komentarza zawiera pięć kluczy wraz z wartościami:

\begin{itemize}[noitemsep]
 \item \emph{comment} --- treść komentarza,
 
 \item \emph{timestamp} --- znacznik czasowy pobierany podczas powstawania komentarza,
 
 \item \emph{history} --- lista zawierająca obiekty o strukturze słownikowej, które przechowują wszelkie zmiany treści komentarza w wartościach kluczy \texttt{comment} oraz \texttt{timestamp}. W momencie zmiany treści przez autora wypowiedzi, dodawany jest nowy element w liście \texttt{history} z obiektem o tych kluczach z wartościami wypełnionymi nową treścią (\texttt{comment}) oraz nowym znacznikiem czasowym (\texttt{timestamp}),
 
 \item \emph{uid} --- autor komentarza,
 
 \item \emph{readby} --- zmienna o strukturze słownika przechowująca informację o użytkownikach, którzy odczytali komentarz i o czasie, w którym to zrobili. W kluczu znajduje się identyfikator autora, natomiast jako wartość wpisywany jest znacznik czasowy odczytania posta.
\end{itemize}

Po zatwierdzeniu zmiany wyświetlana jest najnowsza treść komentarza, natomiast znacznik czasowy pozostaje bez zmian. Taka implementacja jest wymuszona przez konieczność zachowania ciągłości logicznej konwersacji --- edytowany post nie może przemieszczać się w chronologicznym porządku wymiany opinii, ponieważ część wypowiedzi może zostać źle odczytana w przypadku modyfikacji komentarzy przed nią. Intencje edycji zazwyczaj polegają na korygowaniu napisanych wcześniej zdań (interpunkcja, ,,literówki'', ortografia) lub dodawaniu mniej znaczących faktów --- rzadko zdarza się całkowita modyfikacja treści, a w przypadku gdy jednak wystąpi, użytkownicy mogą przejrzeć historię komentarzy, gdy dostrzegą logiczne niezgodności w przepływie opinii.

\subsubsection*{Konflikty}

Struktura pliku z komentarzem może być bardzo dynamiczna i często zmieniana, ponieważ istnieje lista (\texttt{history}), w której dodawane sa elementy, oraz słownik (\texttt{readby}) uzupełniany o kolejne pary klucz-wartość. Czy wobec tego nie generuje to konfliktów, które mogłyby naruszyć spójność danych? Czy użytkownik \texttt{A} może znaleźć się w sytuacji, że ,,ominie'' go część informacji, gdy użytkownik \texttt{B} nadpisze swoją część zamiast części użytkownika \texttt{C}? Lub czy jest możliwość, że modyfikacja użytkownika \texttt{A} zostanie pominięta, ponieważ w tym samym czasie użytkownik \texttt{B} dokonał zmian w tym samym komentarzu i ,,wygrał'' synchronizację?

Odpowiedź brzmi: TAK, jest taka możliwość.

Modyfikacja komentarzy zachodzi w dwóch przypadkach: edycji treści oraz odczytaniu go przez użytkownika. Poniżej zostaną rozważone obydwa przypadki i w każdym z nich opisane zostaną możliwości konfliktów oraz ich skutki.

\begin{description}[noitemsep]
  \item[Modyfikacja treści] --- wystąpienie konfliktu w przypadku edytowania treści jest niemożliwe, ponieważ interfejs użytkownika pozwala na poprawienie treści tylko w przypadku, gdy jesteś autorem komentarza. Jeden użytkownik nie jest w stanie dokonać edycji dwóch postów jednocześnie.
  
  \item[Odznaczenie jako przeczytane] --- w tym przypadku konflikt może wystąpić, lecz nie będzie miał żadnych poważnych konsekwencji. Żeby to wyjaśnić, załóżmy, że użytkownicy \texttt{A} oraz \texttt{B} nie przeczytali jeszcze pewnego komentarza \texttt{x} i zrobią to w tym samym momencie (albo każdy z nich zrobi to na swojej lokalnej wersji, gdy żaden z pozostałych użytkowników należących do folderu nie będzie dostępny). W obecnej chwili (przed wzajemną synchronizacją) obydwaj użytkownicy mają oznaczony post \texttt{x} jako przeczytany. W momencie synchronizacji ,,wygra'' wersja pliku któregoś z użytkowników. Załóżmy, że wygra użytkownik \texttt{A}: wówczas użytkownik \texttt{B} zobaczy w swojej aplikacji \emph{TeamSync}, że ponownie ma nieprzeczytany komentarz, ale wówczas znowu aplikacja ,,przeczyta'' komentarz ponownie odpowiednio go modyfikując (wstawiając nową parę klucz-wartość do słownika \texttt{readby}).
  
  Nawet w przypadku częstych konfliktów, wpływa to nieznacznie tylko i wyłącznie na wygodę użytkownika (ponowne ,,kliknięcie'' na nieprzeczytany wątek).
\end{description}

\subsubsection*{Przykładowy komentarz}

Na rysunku \ref{rys:comment} przedstawiona została struktura komentarza, który dla lepszego zobrazowania własnej struktury został odczytany przez trzech użytkowników i był modyfikowany dwukrotnie.

\begin{figure}[t]
  \label{rys:comment}
  \begin{verbatim}
            {
                "comment": "Ostateczna treść komentarza", 
                "timestamp": "1439395604000", 
                "history": [
                    {
                        "comment": "Pierwsza treść", 
                        "timestamp": "1439395604000"
                    },
                    {
                        "comment": "Pierwsza treść zmodyfikowana", 
                        "timestamp": "1439395945000"
                    },
                    {
                        "comment": "Ostateczna treść komentarza", 
                        "timestamp": "1439396291000"
                    }
                ], 
                "uid": "2HI7KRUNSSONIUJKMRWXGOTIZSHBGFIH", 
                "readby": {
                    "2HI7KRUNSSONIUJKMRWXGOTIZSHBGFIH": "1439395604000",
                    "2SAFDSBCXGFD765GDFS4GFDS35FDSBVB": "1439395888000",
                    "3REW54GFDS6578FDSGDF5BSH652F2B34": "1439396782000"
                }
            }
  \end{verbatim}
  \caption{Struktura przykładowego komentarza.}
\end{figure}

W poniższym przykładzie dobrze widać zmianę treści komentarza poprzez pobranie jej z najświeższej modyfikacji oraz niezmienność znacznika czasowego (\texttt{timestamp}) niezależnie od tego, czy komentarz był modyfikowany. Zauważalna jest też łatwość, z jaką można odtworzyć całą historię zmian wypowiedzi. Oglądanie zmian wszystkich komentarzy jest dostępne dla każdego z użytkowników niezależnie od przynależności, o ile taka historia dla konkretnego przypadku istnieje.

Dzięki słownikowi \texttt{readby} można odczytać, który z użytkowników i w jakim czasie odczytał ten komentarz: na przykład użytkownik o identyfikatorze \texttt{2HI7KRUN\-SSONIUJ\-KMRWXGOTI\-ZSHBGFIH} podczas odczytywania wypowiedzi umieścił swój wpis, pobierając znacznik czasowy \texttt{1439395604000}, co odpowiada dacie \texttt{12.08.2015r.} i godzinie \texttt{18:06} czasu polskiego.

\subsection{Wątki}

Wewnątrz każdego folderu z wątkiem oprócz plików z komentarzami znajduje się plik o nazwie \texttt{meta}, który zawiera dane związane z wątkiem prezentowane w interfejsie użytkownika. Podobnie jak w przypadku komentarzy, są one ustrukturyzowane jako para klucz/wartość w formacie JSON. Klucze znajdujące się w pliku \texttt{meta} obejmują następujące informacje:

\begin{itemize}[noitemsep]
  \item \emph{uid} --- autor wątku,
  \item \emph{timestamp} --- znacznik czasowy pobrany podczas tworzenia wątku (identyczny ze znacznikiem czasowym pierwszego komentarza),
  \item \emph{fileabout} --- ścieżka wewnątrz folderu wskazująca na plik, którego dotyczy watek,
  \item \emph{topic} --- tytuł watku.
\end{itemize}

Plik \texttt{meta} został wprowadzony w celu szybszego odczytywania listy wątków --- podczas przygotowywania listy, odczytywanie przez aplikację komentarzy jest niepotrzebne, dzięki czemu aplikacja działa szybciej. Pobieranie komentarzy z folderu zawierającego wątek (treści, znaczników czasowych itd.) następuje \emph{po} wybraniu konkretnej dyskusji.

Informacje dodatkowe wątku (np. ilość napisanych komentarzy, znacznik czasowy najświeższego komentarza) są na bieżąco obliczane przez aplikację podczas pobierania listy wątków. Znaczniej łatwiej (oraz szybciej) byłoby zapisać te dane w pliku \texttt{meta} i odczytywać je z otwartego już pliku. Natomiast wówczas pojawiłaby się możliwość wystąpienia konfliktów podczas jego edycji. Węzeł, który dodawałby nowy komentarz, musiałby edytować plik \texttt{meta} i inkrementować licznik komentarzy oraz zmodyfikować czas najświeższego komentarza. W przypadku dwóch użytkowników dodających komentarz w tym samym czasie dochodziłoby do konfliktów i umieszczania fałszywych danych wewnątrz pliku.

Dwie z informacji o wątku --- autor oraz znacznik czasowy --- są powielone z pierwszego komentarza w dyskusji (ponieważ podczas zakładania wątku umieszczany jest również inicjujący go komentarz). Zaimplementowano to w taki sposób, aby mieć dostęp do danych o wątku bez konieczności odczytywania któregokolwiek z komentarzy (np. odczytanie pierwszego komentarza wymaga sortowania, ponieważ funkcja modułu \texttt{os} języka \emph{Python} o nazwie \texttt{listdir} listuje nieuporządkowaną zawartość katalogu przyjmowanego w argumencie.

\subsubsection*{Przykładowy plik \texttt{meta}}

Poniżej przedstawiony został przykładowy plik z metadanymi wątku o nazwie ,,Przykładowy wątek''.

\begin{figure}[htb]
  \label{thread}
  \begin{verbatim}
                  {
                      "topic": "Przykładowy wątek", 
                      "timestamp": "1439395604000", 
                      "fileabout": "/katalog1/test.txt",
                      "uid": "2HI7KRUNSSONIUJKMRWXGOTIZSHBGFIH", 
                  }
  \end{verbatim}
  \caption{Struktura przykładowego pliku z metadanymi wątku.}
\end{figure}

Według tego przykładowego pliku wątek został utworzony --- zgodnie ze znacznikiem czasowym --- 30. sierpnia 2015 r. o godzinie \texttt{9:23}. Autorem jest użytkownik o identyfikatorze \texttt{2HI7KRU\-NSSONIUJ\-KMRWXGOTI\-ZSHBGFIH}. Dyskusja o tytule ,,Przykładowy wątek'' rozpoczęta przez tego użytkownika dotyczy pliku \texttt{test.txt}, który znajduje się wewnątrz folderu \texttt{katalog1} wewnątrz folderu synchronizowanego.

Strukturę przykładowego synchronizowanego folderu --- zawierający zarówno wątek z powyższego przykładu jak i plik opisywany w dyskusji --- przedstawiono poniżej:

\begin{minipage}{\linewidth}
\vspace{1em}
\begin{verbatim}
    [root]/
        .sync/
            [ pliki aplikacji BitTorrent Sync ]
        .Users/
            [ pliki z danymi o użytkownikach ]
        .Comments/
            katalog1/
                1439395604000@#&$Przykładowy wątek/
                    meta
                    1439395604000@#&$2HI7KRUNSSONIUJKMRWXGOTIZSHBGFIH
            1439423576000@#&$Inny wątek/
                meta
                1439423576000@#&$2SAFDSBCXGFD765GDFS4GFDS35FDSBVB
                1439476598000@#&$2HI7KRUNSSONIUJKMRWXGOTIZSHBGFIH
        katalog1/
            test.txt
            ccc.txt
        aaa.txt
        bbb.txt
\end{verbatim}
\vspace{1em}
\end{minipage}

Widać wewnątrz poniższej struktury, że ,,Przykładowy wątek'' został umieszczony w folderze \texttt{.Comments} w lokalizacji takiej samej, jak oryginalny plik, o którym powstała dyskusja (\texttt{katalog1}). Wewnątrz niego znajduje się plik z metadanymi o wcześniej przedstawionej strukturze oraz jeden komentarz. Warto zauważyć, że w nazwie komentarza jest ten sam znacznik czasowy, który jest zapisany w pliku \texttt{meta} --- według zasad opisanych wcześniej.

W przykładzie umieszczono również drugi wątek, zlokalizowany najpłycej w katalogu \texttt{.Comments}. Oznacza to, że w interfejsie graficznym będzie on wylistowany zawsze, gdy użytkownik nawigując po folderze, będzie przebywał na najwyższym jego poziomie --- ,,korzeniu''. ,,Inny wątek'' --- bo tak należy odczytać tytuł tej dyskusji, zawiera dwa komentarze --- jeden napisany przez użytkownika \texttt{2SAFDSB\-CXGFD765\-GDFS4GFDS\-35FDSBVB} w chwili pobrania znacznika czasowego \texttt{1439423\-576000} oraz przez użytkownika \texttt{2HI7KRU\-NSSONIUJ\-KMRWXGO\-TIZSHBGFIH} w chwili \texttt{1439476\-598000}.

\subsection{Nazwy folderów i plików}

\label{filenamesf}

Foldery zawierające wątki oraz pliki z komentarzami są tworzone przez dowolny węzeł w dowolnym czasie, co zwiększa niebezpieczeństwo wystąpienia konfliktów w nazwach tworzonych plików za każdym razem, gdy użytkownik wypowiada się w dyskusji lub rozpoczyna nową. Dodatkowo drzewiasta struktura katalogu \texttt{.Comments}, wewnątrz którego przechowywane są komentarze, powoduje, że ,,zwykłe'' foldery przechowywane są obok katalogów zawierających pliki komentarzy. Przeszukiwanie ich wszystkich przez serwer poprzez przeglądanie wszystkich lokalizacji jest wolniejsze i przy folderze mocno rozbudowanym systemem katalogów może odczuwalnie spowolnić aplikację.

Mając na uwadze powyższe problemy, projektując aplikację \emph{TeamSync} przyjęto następujące dwie zasady --- dotyczące nazw zarówno plików komentarzy, jak i folderów z wątkami --- jako konieczne do spełnienia. Konsekwencje nieprzestrzegania każdej z zasad będą wyjaśnione w dalszej części sekcji.

\begin{enumerate}[noitemsep]
  \item System powinien uniemożliwiać (zapewniać poziom prawdopodobieństwa \emph{w praktyce} uniemożliwiający) powtarzanie się \emph{nazwy} pliku komentarza lub folderu wątku.
  
  \item Nazwy powinny zawierać w sobie pewien element (ciąg znaków), dzięki któremu serwer odczyta informację, że wybrany folder lub plik to odpowiednio folder zawierający wątek oraz plik przechowujący wypowiedź. Informacja o wyjątkowości katalogu musi być zawarta w jego nazwie.
\end{enumerate}

Opisy sposobów realizacji obydwóch zasad zostały opisane w poniższych podrozdziałach.

\subsubsection*{Unikalność nazwy}

Jeśli w przypadku plików z komentarzami unikalność wprowadzanych nazw nie byłaby zachowana, efektem byłby konflikt skutkujący nadpisaniem pliku. Użytkownik \texttt{A} zatwierdzający treść odpowiedzi umieściłby swój plik, który chwilę później byłby nadpisany przez użytkownika \texttt{B}. Wersja użytkownika \texttt{A} zostałaby uznana jako nieaktualna i w konsekwencji przeniesiona do katalogu \texttt{Archive} w ukrytym folderze \texttt{.sync}, do którego użytkownicy nie mają dostępu z poziomu aplikacji \emph{TeamSync}.

Podobnie --- lecz z innymi konsekwencjami --- wygląda sytuacja w przypadku tworzenia nowego wątku. Gdyby ten sam użytkownik \texttt{A} tworzył nowy wątek w tym samym momencie, co użytkownik \texttt{B}, również wystąpiłby konflikt, lecz z odmiennymi skutkami. W kodzie serwera zastosowano instrukcję warunkową przed tworzeniem nowego folderu wątku sprawdzającą, czy taki folder już istnieje. Aplikacja tworzy nowy katalog, jeśli jeszcze nie istniał wcześniej w systemie plików --- w odwrotnym przypadku serwer kontynuuje działanie bez żadnych zmian. Efektem konfliktu byłoby nienadpisanie folderu z wątkiem, a nadpisanie pliku \texttt{meta} oraz (w zależności czy unikalność nazwy pliku z komentarzem została zachowana) możliwość nadpisania lub dodania nadmiarowego komentarza. W obecnym przypadku (użytkownicy \texttt{A} oraz \texttt{B} próbujący dodać nowy wątek w tym samym momencie) dodanie komentarza mogłoby skutkować następującą zawartością folderu z wątkiem:

\begin{itemize}[noitemsep]
  \item plik \texttt{meta} z treścią umieszczoną przez zwycięzcę konfliktu \texttt{A} vs. \texttt{B},
  
  \item komentarz użytkownika \texttt{A},
  
  \item komentarz użytkownika \texttt{B}.
\end{itemize}

Jak widać zapewnienie unikalności nazwy plików z komentarzami oraz folderów z wątkami w kontekście działania najważniejszej funkcjonalności jest niezbędne. Dlatego uzyskanie jednoznacznych nazw w przypadku plików z komentarzami odbywa się dzięki następującym składnikom:

\begin{itemize}[noitemsep]
  \item znacznik czasowy pobrany z serwera NTP podczas pisania nowego komentarza z dokładnością do jednej sekundy,
  
  \item identyfikator autora wypowiedzi (\emph{uid}).
\end{itemize}

Dzięki kombinacji tych dwóch składników nie ma możliwości uzyskać w systemie identycznej pary nazw pliku z komentarzem. Żeby złamać zasadę i uzyskać taką parę, ten sam użytkownik musiałby wysłać dwa komentarze w ciągu jednej sekundy, co --- ze względu na długość przetwarzania umieszczania odpowiedzi w systemie plików, oczekiwanie na odpowiedź ze znacznikiem czasowym z serwera NTP i otrzymania odpowiedzi przez klienta w przeglądarce --- jest bardzo mało prawdopodobne.

Nawet jeśliby tak się stało, druga wysłana przez użytkownika wypowiedź byłaby pusta, ponieważ nie miałby czasu na wypełnienie jej treścią --- interfejs graficzny automatycznie po zatwierdzeniu komentarza usuwa całą treść z pola tekstowego i jest gotowy do umieszczania następnej wypowiedzi. W momencie odnotowania przez serwer faktu, że komentarz ma pustą treść, przekaże błąd użytkownikowi za pomocą komunikatu: ,,\emph{Treść komentarza nie może być pusta}''. Użytkownik musiałby bardzo szybko umieścić dowolną treść wewnątrz komentarza i ponownie zatwierdzić umieszczanie wypowiedzi przyciskiem ,,\emph{Odpowiedz}''. Jednak --- jak zostało napisane powyżej --- jest to bardzo mało prawdopodobne.

Przykładowa nazwa pliku z komentarzem mogłaby być następująca:

\begin{center}
  \texttt{1439939641000<<separator>>2HI7KRUNSSONIUJKMRWXGOTIZSHBGFIH}
\end{center}

Komentarz został wprowadzony do systemu w chwili o znaczniku czasowym równym \texttt{14399\-39641} (\texttt{000} na końcu znacznika są dodane w wyniku jego konwersji z jednostek sekund na milisekundy) przez użytkownika o identyfikatorze \texttt{2HI7KRUN\-SSONIUJK\-MRWXGOTI\-ZSHBGFIH}. Łańcuch znaków \texttt{<<separator>>} zostanie omówiony w późniejszej części podrozdziału.

Z tych samych powodów, dla których nazwy plików z komentarzami łączone są z dwóch powyżej opisanych składników (identyfikator użytkownika oraz znacznik czasowy), zdecydowano się otrzymywać w taki sposób również nazwy katalogów zawierających treść dyskusji. Dodatkowo system \emph{TeamSync} uniemożliwia zatwierdzenie nowego wątku z pustym polem tekstowym zawierającym tytuł dyskusji --- jest to dodatkowe ,,utrudnienie'' umieszczania dwóch wątków w tej samej sekundzie.

\subsubsection*{Separator nazwy}

\label{filenames}

Ciąg znaków zapisywany we wcześniejszych przykładach jako \texttt{separator} wchodzi w skład nazwy zarówno plików z komentarzami, jak i folderów z wątkami. Dzięki jego obecności w nazwie serwer przeszukując folder \texttt{.Comments} rozróżnia foldery pełniące funkcję przechowywania wątków od tych, które służą do grupowania wątków w strukturę będącą odwzorowaniem struktury folderu synchronizowanego.

Ponieważ opisywany łańcuch znaków (opisywany dalej jako \emph{separator}) jest częścią nazwy pliku, musi składać sie ze znaków akceptowanych przez system operacyjny. Na przykład w systemach operacyjnych z rodziny \emph{Unix}, w separatorze nie mógłby się znaleźć znak \texttt{/}, ze względu na pełnioną funkcję w systemie plików podczas określania lokalizacji.

Podstawowym problemem w implementacji związanym z nazwami plików i katalogów ,,specjalnych'' (tych z wątkami oraz komentarzami) był fakt, że skoro użytkownicy mogą dodawać pliki o dowolnych nazwach, to mogą dodać też plik zawierający w nazwie separator. Wówczas serwer czytając listę wątków lub próbując przeczytać jeden z komentarzy, zwróciłby błąd. Ze względu na brak możliwości wprowadzenia takiego separatora, aby użytkownik nie mógł go --- na drodze przypadku lub nie --- umieścić w nazwie ,,zwykłego'' pliku, zdecydowano się na zminimalizowanie prawdopodobieństwa wystąpienia podobnej sytuacji.

Separator --- w obecnej implementacji --- ma postać ciągu czterech znaków \texttt{@\#\&\$}, a jego postać została wybrana ze względu na bardzo mało realną szansę zawarcia takiego ciągu znaków w nazwie pliku przez użytkowników. W ostatecznej postaci plik z komentarzem z poprzedniej sekcji zapisany w systemie plików użytkownika mógłby wyglądać następująco:

\begin{center}
  \texttt{1439939641000@\#\&\$2HI7KRUNSSONIUJKMRWXGOTIZSHBGFIH}
\end{center}

Jako dodatkowe zabezpieczenie można potraktować fakt, że serwer podczas przeszukiwania folderu \texttt{.Comments} w poszukiwaniu folderów z wątkami przyrównuje nazwę plików do wzorca wyrażenia regularnego:

\begin{center}
  \texttt{[0-9]\{13\} @\#\&\textbackslash\$ [A-Z0-9]\{32\}}
\end{center}

Aby serwer odczytał fałszywy folder z wątkiem (czyli ten wprowadzony przez użytkownika --- niezawierający ani pliku \texttt{meta}, ani plików z komentarzami), użytkownik musiałby stworzyć folder (albo plik) o nazwie pasującej do powyższego wzorca. Podczas rzetelnego korzystania z aplikacji \emph{TeamSync} \emph{w praktyce} niemożliwe jest spowodowanie błędu.

Użytkownicy chcący zaburzyć pracę działania programu, są w stanie to osiągnąć, nazywając folder w podobny sposób (według wzorca zaprezentowanego wcześniej wyrażenia regularnego), umieszczając go w dowolnej lokalizacji wewnątrz katalogu synchronizowanego i tworząc w nim wątek. Wówczas --- podczas tworzenia wątku --- serwer umieści wewnątrz katalogu \texttt{.Comments} folder o nazwie spełniającej warunki folderu ,,specjalnego'', a będącego nim --- brak będzie plików z komentarzami oraz pliku \texttt{meta}. Zabezpieczenie aplikacji w takim wypadku polega na ignorowaniu folderów, które spełniają kryterium nazwy, lecz nie zawierają w sobie pliku z metadanymi wątku.

\section{Serwer NTP}

\begin{figure}[htb]
  \vspace{5pt}
  \begin{center}
    \includegraphics[width=230pt]{figures/architecturentp.png}
  \end{center}
  \caption{Komunikacja systemu \emph{TeamSync} z serwerem NTP.}
\end{figure}

Istnienie możliwości zamieszczania komentarzy przez użytkowników w dowolnej chwili działania systemu \emph{TeamSync} oraz konieczność zachowania informacji o czasie wprowadzania wypowiedzi są źródłami kilku problemów. Chcąc umożliwić rzetelne --- nie tylko co do treści, ale również co do czasu zamieszczenia komentarza --- dzielenie się swoimi opiniami przez użytkowników, należało uniknąć w systemie dwóch nastepujących przeszkód:

\begin{itemize}[noitemsep]
  \item nie ma pewności, że na każdym z węzłów jest ten sam czas systemowy, z którego użytkownik pobierałby znacznik czasowy podczas umieszczania komentarza. Jeśli różnica między czasami zamieszczanych komentarzy byłaby większa niż kilka minut, wyświetlane wypowiedzi często byłyby przedstawiane w nieprawidłowej kolejności. Komentarze napisane później (według czasu rzeczywistego) --- w wyniku różnic czasów systemowych na węzłach --- mogłyby zostać umieszczone w sieci jako napisane wcześniej niż w rzeczywistości,
  
  \item nie ma gwarancji, że wszystkie węzły w jednolity sposób będą wprowadzać znaczniki czasowe do systemu --- problem użytej przez nich jednostki oraz formatu.
\end{itemize}

\emph{Serwer NTP} \cite{ntp}, od którego użytkownicy pobierają znaczniki czasowe podczas zamieszczania komentarzy, pełni rolę neutralnego arbitra i zarazem stanowi rozwiązanie dwóch powyższych zastrzeżeń. Otrzymywanie znaczników czasowych z tego samego dla wszystkich użytkowników, neutralnego źródła jest kluczowym punktem w algorytmie dodawania nowych wypowiedzi --- w przypadku jego braku, użytkownicy mogliby wprowadzać do sieci fałszywe dane.

Potrzeba zachowania jednolitego formatu oraz jednostki znacznika czasowego wynika z konieczności zapewnienia unikalności nazw plików zawierających dane z komentarzami. Bez dostępu do serwera NTP użytkownik nie ma możliwości napisać wypowiedzi w żadnym z wątków, ponieważ narażałoby to system na generowanie konfliktów (np. poprzez dwukrotne uzyskanie tego samego znacznika czasowego przy dodaniu komentarza).

Zapytanie wysyłane jest do serwera NTP \emph{zawsze}, gdy użytkownik umieści treść komentarza (lub nowego wątku) w interfejsie graficznym i zatwierdzi jego dodanie. W odpowiedzi serwer NTP odsyła wiele wartości takich jak: znacznik czasowy, przesunięcie, precyzja, wersja protokołu itd. Najważniejszym z nich (i jedynym użytym w systemie) jest sam znacznik czasowy.

Znacznikiem czasowym, który użytkownicy otrzymują w odpowiedzi jest skoordynowanym czasem uniwersalnym (\emph{UTC --- Universal Coordinated Time}) w postaci liczby całkowitej zawierającą sekundy.
\input{chapters/05-konfiguracja.tex}
\chapter{Implementacja}

\section{Zastosowane technologie}

Wybór technologii użytych w implementacji został podyktowany jej architekturą --- interfejs użytkownika dostępny jest tylko i wyłącznie w przeglądarce internetowej, z tego powodu wybrane przez autora i opisane poniżej narzędzia są kojarzone jedynie w obszarze technologii internetowych. Interfejs graficzny wywarł niemały wpływ na wybór narzędzi, z uwagi na fakt, że osoba korzystająca z aplikacji powinna w łatwy i intuicyjny sposób:

\begin{itemize}[noitemsep]
  \item poruszać się po katalogach, plikach oraz wątkach,
  \item wyszukiwać wątki, komentarze lub pojedyncze słowa wewnątrz nich,
  \item wydzielić te wątki, które najbardziej interesują użytkownika,
  \item przeglądać statystyki dotyczące wpisywanych komentarzy.
\end{itemize}

Narzędzie, które zostało wybrane do realizacji tych zadań to AngularJS \cite{angular} --- technologia ułatwiająca kontrolę nad elementami interfejsu, stworzona w języku JavaScript. Jednakże, aby cały system prawidłowo i sprawnie działał, potrzebna była również technologia, która umożliwia przekazanie do interfejsu danych w odpowiedniej formie, pobranie z niego zadań wyznaczonych przez użytkownika i odpowiednie ich zrealizowanie. Do tych celów użyto języka Python \cite{python} wraz z narzędziem o nazwie Django \cite{django} do implementacji aplikacji internetowych. Połączenie Python/Django oraz AngularJS okazało się wystarczające do stworzenia aplikacji --- poza nimi użyte zostały wyłącznie skrypty odpowiadające za wygląd interfejsu.

\subsection*{AngularJS}

AngularJS jest otwartą biblioteką języka JavaScript, która została stworzona w 2009 roku i obecnie jest wspierana przez firmę Google. Najważniejszą funkcją narzędzia jest dwukierunkowe wiązanie danych (tzw. ,,emph{two-way data binding}''), które wyjątkowo łatwo obsługuje się wewnątrz szablonów HTML --- za ich pomocą zmiany wewnątrz modelu są natychmiast odzwierciedlone w interfejsie i na odwrót: elementy wprowadzone bądź zmienione przez użytkownika natychmiastowo znajdują odzwierciedlenie w modelu.

Biblioteka korzysta z wzorca MVC (Model-View Controller), aby ułatwić nie tylko implementację, ale również testowanie tworzonego systemu. AngularJS umożliwia bardzo łatwe zarządzanie dynamiczną treścią poprzez wspomniane wcześniej szablony i dodatkowe komendy wewnątrz szablonów. Dynamika aplikacji zaimplementowanej w niniejszej pracy była głównym wyzwaniem podczas jej tworzenia, a biblioteka AngularJS znacząco wpłynęła zarówno na poprawność, jak i objętość kodu.

\subsection*{Django}

O ile AngularJS został wykorzystany w znacznej mierze, o tyle stopień wykorzystania funkcjonalności Django został zredukowany do minimum. Biblioteka Django jest biblioteką do kompleksowego budowania stron internetowych w dowolnych architekturach (np. REST \cite{rest}), stworzoną, by jak najbardziej automatyzować kod i jak najszybciej --- a zarazem jak najprościej --- tworzyć skomplikowane serwisy. Narzędzie użyte zostało do implementacji z uwagi na prostotę i możliwości testowania, debugowania oraz możliwości rozbudowania zaimplementowanej aplikacji o dodatkowe rozwiązania.

Biblioteka Django powstała w 2003 roku, stworzona została przez programistów związanych ze środowiskiem dziennikarskim, dzięki czemu przystosowana jest do szybkiej i nieskomplikowanej pracy. Podobnie jak AngularJS pozwala (poprzez mechanizm szablonów) na łatwe i intuicyjne umieszczanie dynamicznych elementów na stronie. Wybrana została głównie ze względu na swoją prostotę oraz dodatkowe funkcjonalności ułatwiające pracę nad aplikacją np. możliwość automatycznego ponownego uruchamiania serwera podczas wprowadzania w nim zmian (co jest robione automatycznie).

\section{Komunikacja z serwerem}

Komunikacja między klientem a serwerem odbywa się poprzez zapytania \texttt{HTTP}, wewnątrz których przesyłane są dane potrzebne zarówno stronie klienta do ich wyświetlenia, jak i stronie serwera do wprowadzenia ich do systemu. Poniżej dokładnie opisane zostaną operacje wykonywane przez serwer, podzielone na grupy ze zbliżonym obszarem funkcjonalności.

\subsection{Foldery}

Operacje dostępne dla folderów w aplikacji TeamSync zostały ograniczone do dodawania ich oraz usuwania. Dodatkowe okna z ustawieniami, edycja folderów zostały przewidziane jako ewentualną dalszą rozbudowę systemu. W pracy skupiono się na pełnowartościowym korzystaniu z podstawowej funkcjonalności (synchronizacja danych oraz wprowadzanie komentarzy), co zostało spełnione bez konieczności implementacji edycji folderów.

\subsubsection*{Pobieranie listy folderów}

\begin{figure}[h!]
  \vspace{5pt}
  \begin{center}
    \includegraphics[width=250pt]{figures/metgetfolders.png}
  \end{center}
  \caption{Komunikacja w systemie \emph{TeamSync} podczas wywoływania metody \texttt{getfolders} --- pobieranie listy folderów.}
  \label{picmetgetfolders}
\end{figure}

Aby umożliwić użytkownikowi wybór folderu z listy synchronizowanych katalogów, aplikacja przeglądarkowa wysyła do serwera zapytanie \texttt{HTTP} za pomocą metody \texttt{GET} na adres \texttt{http://<adres oraz port serwera TeamSync>\-/getfolders}. Odpowiedź z serwera umieszczana jest wewnątrz listy \texttt{folders}, która przechowuje informacje dotyczące katalogów zapisane w formacie \texttt{JSON}.

\begin{figure}[htb!]
\label{newcommentrequest}
  \begin{verbatim}
          [
              {
                  "name": "testowy katalog", 
                  "dir": "/home/user/testowy katalog", 
              
                /* wartości mniej istotne, pobrane podczas
                   wykonywania metody get_folders takie jak: size,
                   down_speed, up_speed, error, indexing, id, type */
         
                  "secret": "A3LL43HJ257YCKMOLAD7QSEAS7U373BVO", 
                  "identity": "Filip Rachwalak", 
                  "uid": "FWZPQQKUE3JPQKWY557FERJ4SD3BSFMM", 
                  "users": [
                      {
                          "uid": "PWGGJTDPC35O43SLEKFPBPIG3NYV7PH7", 
                          "identity": "Jan Iksiński"
                      }, 
                      {
                          "uid": "FWZPQQKUE3JPQKWY557FERJ4SD3BSFMM", 
                          "identity": "Filip Rachwalak (Ty)"
                      }
                  ]
              }
          ]
  \end{verbatim}
  \caption{Przykładowa odpowiedź serwera na żądanie pobrania listy synchronizowanych folderów.}
\end{figure}

W odpowiedzi serwera znajdują się wszystkie informacje dotyczące folderów otrzymane z BitTorrent Sync API (w sekcji \ref{btsyncapiproto} znajduje się dokładne wyjaśnienie zwracanych danych przez metodę \texttt{get\_folders}) z dodatkowymi danymi:

\begin{itemize}[noitemsep]
  \item \emph{name} --- ułatwia interfejsowi wyświetlanie nazwy katalogu w liście,
  
  \item \emph{identity} oraz \emph{uid} --- tożsamość i identyfikator użytkownika,
  
  \item \emph{users} --- lista użytkowników zawierająca ich identyfikatory oraz tożsamości, potrzebna do wyświetlania wszystkich użytkowników w folderze oraz autorów komentarzy. Wartości wewnątrz tej listy są pobrany z katalogu \texttt{.Users} wewnątrz folderu współdzielonego.
\end{itemize}

W powyższym przykładzie użytkownik o identyfikatorze \texttt{FWZPQQKU\-E3JPQKWY\-557FERJ4\-SD3BSFMM} i tożsamości \texttt{Filip Rachwalak} posiada tylko jeden współdzielony folder, który dzieli z użytkownikiem \texttt{Jan Iksiński} o identyfikatorze \texttt{PWGGJTDP\-C35O43SL\-EKFPBPIG3\-NYV7PH7}. Bezpośrednia ścieżka katalogu to \texttt{/home/user/\-testowy katalog}, natomiast nazwa, jaka będzie wyświetlana na liście w przeglądarce to \texttt{testowy katalog}. \emph{Secret} folderu to \texttt{A3LL43HJ2\-57YCKMOL\-AD7QSEAS7\-U373BVO}.

\subsubsection*{Tworzenie folderu}

\begin{figure}[h!]
  \vspace{5pt}
  \begin{center}
    \includegraphics[width=230pt]{figures/metaddfolder.png}
  \end{center}
  \caption{Komunikacja w systemie \emph{TeamSync} podczas wywoływania metody \texttt{addfolder} --- tworzenie folderu.}
  \label{picmetgetfolders}
\end{figure}

Użytkownik ma możliwość wyboru pomiędzy rodzajem dodawanego katalogu --- inicjacja własnego lub dołączenie do już istniejącego, stworzonego przez innego użytkownika folderu. Podczas obydwóch tych operacji przeglądarkowy interfejs graficzny wysyła do serwera zapytanie metodą \texttt{POST} na adres \texttt{http://<adres oraz port serwera TeamSync>\-/addfolder}. W parametrach żądania znajdują się:

\begin{itemize}[noitemsep]
  \item \emph{path} --- bezwzględna ścieżka do folderu, którego zawartość od tej pory będzie synchronizowana z pozostałymi użytkownikami,
  
  \item \emph{identity} --- zadeklarowana przez użytkownika tożsamość, czyli nazwa użytkownika, jaka będzie wyświetlana innym węzłom,
  
  \item \emph{secret} --- za jego pomocą BitTorrent Sync odnajdzie w sieci pozostałych użytkowników; jeśli ten parametr pozostanie pusty, serwer stworzy nowy folder (nie w fizycznym sensie, lecz w logicznym --- doda podany w parametrze \emph{path} folder do folderów współdzielonych).
\end{itemize}

Serwer otrzymując żądanie z powyższymi parametrami, przekazuje je poprzez zapytanie \texttt{HTTP} do BitTorrent Sync API, za pomocą funkcji \texttt{add\_folder} (dokładny opis metody znajduje się w sekcji \ref{btsyncapiproto}). Jeśli podstawowa walidacja wprowadzanych danych się nie powiedzie, lub jeśli BitTorrent Sync API zwróci błąd, serwer oprócz zwrócenia odpowiedzi z treścią błędu nie wykona żadnych dodatkowych instrukcji.

W przypadku powodzenia serwer musi wykonać następujące czynności:

\begin{description}[noitemsep]
  \item[Utworzenie folderów \texttt{.Comments} oraz \texttt{.Users}] --- jeśli użytkownik nie dołącza do istniejącego folderu, tylko inicjuje swój katalog, musi stworzyć lokalizacje, do których pozostali użytkownicy będą mogli dopisywać pliki ze swoimi danymi (\texttt{.Users}) oraz komentarze (\texttt{.Comments}).
  
  \item[Uaktualnienie folderu \texttt{.Users}] --- aby inni użytkownicy mogli zobaczyć nowy węzeł w swoim folderze, użytkownik dołączający do katalogu musi umieścić w folderze \texttt{.Users} swoje dane --- identyfikator oraz tożsamość.
  
  \item[Uaktualnienie pliku konfiguracyjnego] --- do słownika \texttt{identities} --- przechowującego wszystkie tożsamości użytkownika --- w pliku konfiguracyjnym (\texttt{config.json}), dodawana jest nowa tożsamość.
\end{description}

Po wykonaniu powyższych czynności serwer zwraca komunikat do aplikacji klienckiej o powodzeniu operacji, a w przeglądarce odświeżana jest lista folderów.

\subsubsection*{Usuwanie folderu}

\begin{figure}[h!]
  \vspace{5pt}
  \begin{center}
    \includegraphics[width=240pt]{figures/metdeletefolder.png}
  \end{center}
  \caption{Komunikacja w systemie \emph{TeamSync} podczas wywoływania metody \texttt{addfolder} --- tworzenie folderu.}
  \label{picmetdeletefolder}
\end{figure}

Usuwanie katalogu odbywa się poprzez wysłanie do serwera żądania \texttt{HTTP} za pomocą metody \texttt{POST} na adres \texttt{http://<adres oraz port serwera TeamSync>\-/deletefolder}. Parametrem przesyłanym wewnątrz zapytania jest \emph{secret} folderu. Serwer nie robi nic poza przesłaniem \emph{secreta} za pomocą funkcji \texttt{remove\_folder} protokołu BitTorrent Sync API (dokładny opis funkcji \texttt{remove\_folder} w sekcji \ref{btsyncapiproto}) i  przekazuje odpowiedź --- o pomyślnym lub niepomyślnym usunięciu --- do klienta.

W przypadku niepowodzenia rola serwera kończy się na przekazaniu wyniku. Natomiast jeśli usunięcie folderu z listy synchronizowanych katalogów przebiegnie pomyślnie, serwer musi zmodyfikować plik konfiguracyjny \texttt{config.json} w celu usunięcia ze zmiennej \texttt{identities} tożsamości, której użytkownik już nie będzie potrzebował, ponieważ folder stanowiący jej środowisko przestał istnieć.

\subsection{Komentarze}

\label{comments}

Aplikacja TeamSync nie umożliwia usuwania komentarzy przez użytkowników nawet w przypadku gdy osobą, która chciałaby usunąć wypowiedź z systemu, jest jej autor. Usunięcie komentarza można zastąpić poprzez zmodyfikowanie lub całkowite usunięcie jego treści, natomiast obecność wypowiedzi w systemie będzie zawsze widoczna.

\subsubsection*{Czytanie komentarzy}

\begin{figure}[h!]
  \vspace{5pt}
  \begin{center}
    \includegraphics[width=320pt]{figures/metgetcomments.png}
  \end{center}
  \caption{Komunikacja w systemie \emph{TeamSync} podczas wywoływania metod \texttt{getcomments}, \texttt{getallcomments} oraz \texttt{getcommentsfrompath} --- czytanie komentarzy.}
  \label{picmetdeletefolder}
\end{figure}

Główną czynnością aplikacji TeamSync podczas wyświetlania komentarzy jest pobranie ich z serwera i przeniesienie odpowiedzi serwera do listy \texttt{comments}, która jest główną strukturą wyświetlaną w sekcji komentarzy. Zawiera ona komentarze w takiej strukturze, w jakiej są one przeczytane przez serwer z plików wewnątrz folderu z wątkiem.

Pobieranie listy komentarzy inicjowane jest z kodu \texttt{javascript} przeglądarki, a użyte do tego celu zapytania różnią się w zależności od widoku (aby zapoznać się dokładniej z widokami, należy przeczytać dodatek \ref{views}), w którym użytkownik aktualnie się znajduje:

\begin{description}[noitemsep]
  \item[W widoku nowego wątku] nie są pobierane żadne komentarze. W momencie wprowadzenia danych i zatwierdzenia nowego wątku system automatycznie przechodzi do widoku aktualnego wątku.
  
  \item[W widoku aktualnego wątku] odświeżanie listy komentarzy odbywa się za pomocą funkcji \texttt{re\-freshComments} wywoływanej np. podczas przechodzenia do nowego wątku albo wprowadzenia nowego komentarza. Jej zadaniem jest wysłanie na URL serwera \texttt{http://<adres oraz port serwera TeamSync>\-/getcomments} zapytania metodą \texttt{POST}, wewnątrz którego znajdują się następujące dane:
  \begin{itemize}[noitemsep]
    \item \emph{fullthreadpath} --- pełna ścieżka lokalizacji wątku, niezbędna do pobrania listy komentarzy,
    \item \emph{sortinguid} --- identyfikator użytkownika, według którego mają zostać posortowane komentarze. Jeśli wartość ta pozostanie pusta (domyślnie), serwer posortuje komentarze według ich znaczników czasowych pobranych w momencie umieszczenia ich w systemie TeamSync. W przeciwnym wypadku (\emph{sortinguid} wskazuje któregoś z użytkowników), wypowiedzi zostaną posortowane w takiej kolejności, w jakiej zostały odczytane przez wskazanego w argumencie \emph{sortinguid} użytkownika. Serwer wówczas --- zanim zwróci w odpowiedzi listę komentarzy --- posortuje dane według struktury \texttt{readby}, wewnątrz której umieszczone są znaczniki czasowe momentu odczytania wiadomości przez wskazanego użytkownika (sortowanie komentarzy według różnych spójności zostało szczegółowo opisane w sekcji \ref{consistencies}).
  \end{itemize}
  Jak zostało opisane na początku sekcji --- serwer odpowiada listą komentarzy odczytanych z plików i w zależności od zawartości argumentu \emph{sortinguid} sortuje wyjściowe dane przed ich wysłaniem do klienta.

  \item[W widoku bieżącej lokalizacji] do zmiennej \texttt{comments} zostaną przypisane komentarze z odpowiedzi serwera na zapytanie wysłane przez funkcję \texttt{getCommentsFromPath} uruchamianą w kodzie \texttt{javascript} przeglądarki klienta. Funkcja ta wysyła żądanie metodą \texttt{POST} na adres URL serwera \texttt{http://<adres oraz port serwera TeamSync>\-/getcommentsfrompath} z danymi w formacie JSON:
  \begin{itemize}[noitemsep]
    \item \emph{folderpath} --- bezwzględna ścieżka synchronizowanego folderu.
    \item \emph{insidepath} --- lokalizacja wewnątrz folderu, z której poziomu będą pobierane wszystkie komentarze. Nie będą brane pod uwagę komentarze w wątkach umieszczonych wewnątrz poziomów, wgłąb lokalizacji.
  \end{itemize}
  Konieczność wysłania dwóch ścieżek zamiast jednej podyktowana jest faktem, iż w przypadku pobierania komentarzy z jednego wątku zmienna \emph{fullthreadpath} zawarta jest --- obok innych informacji, np. nazwie, znaczniku czasowym powstania itd. --- w obiekcie wyświetlanym przez graficzny interfejs. Podczas pobierania komentarzy z jednego wątku wystarczy przekazać tą zmienną serwerowi. Natomiast w obecnym wypadku pobierania komentarzy z całej lokalizacji, nie ma jednego konkretnego wątku, z którego można by pozyskać tą informację, więc zostało przyjęte rozwiązanie, w którym klient wysyła dwie ścieżki, a serwer łączy je w całość (łącznie z katalogiem \texttt{.Comments} ścieżki łączone są na wzór \texttt{<folderpath>/.Comments/<insidepath>}) i dopiero z tak uzyskanej ścieżki pobierane są komentarze.
  
  \item[W widoku wszystkich komentarzy] w kodzie skryptu interfejsu uruchamiana jest nie przyjmująca żadnego argumentu funkcja \texttt{get\-All\-Comments}. Jej zadaniem jest wysłanie metodą \texttt{POST} na adres URL serwera \texttt{http://<adres oraz port serwera TeamSync>/getallcomments} bezwzględnej ścieżki synchronizowanego folderu. Serwer po jej odebraniu pobierze wszystkie komentarze napisane wewnątrz katalogu niezależnie od poziomu zagłębienia wewnątrz niego.
  
  \item[W widoku wszystkich komentarzy użytkownika] procedura uzyskiwania od serwera wyświetlanych postów wygląda identycznie jak w przypadku widoku wszystkich komentarzy. Jedyną różnicą w sposobie ich wyświetlania jest filtrowanie ich po stronie klienta według \emph{uid} autora wypowiedzi. Użytkownik wybierając autora, którego komentarze chciałby wyświetlić, w rzeczywistości pobiera wszystkie wypowiedzi z folderu i dopiero po ich uzyskaniu nakładany jest filtr. Dzięki takiemu podejściu możliwe jest szybsze oglądanie komentarzy różnych użytkowników, ponieważ pobierane one są raz. Zmieniając autora, zmieniane jest tylko filtrowanie wypowiedzi.
\end{description}

\subsubsection*{Pisanie komentarzy}

\begin{figure}[h!]
  \vspace{5pt}
  \begin{center}
    \includegraphics[width=250pt]{figures/metwritecomment.png}
  \end{center}
  \caption{Komunikacja w systemie \emph{TeamSync} podczas wywoływania metody \texttt{writecomment} --- pisanie komentarzy.}
  \label{picmetdeletefolder}
\end{figure}

Użytkownik przeglądając komentarze, może odpowiadać na nie tylko w przypadku, gdy przegląda je w widoku wątku (aby zapoznać się dokładniej z widokami, należy przeczytać dodatek \ref{views}). Nie ma możliwości odpisywania odpowiedzi na komentarz, nie znając kontekstu całego wątku, gdyż może to doprowadzić do nieporozumień między użytkownikami. Dlatego też możliwość komentowania we wszystkich widokach oprócz widoku wątku została zablokowana.

W widoku wątku --- na końcu konwersacji po obecnie ostatnim poście --- znajduje się pole tekstowe, gdzie użytkownik może wpisać treść swojej odpowiedzi, a następnie kliknąć przycisk, który metodą \texttt{POST} wysyła na adres \texttt{http://<adres oraz port serwera TeamSync>/\-writecomment} żądanie zawierające obiekt \texttt{JSON} z wartościami o następujących kluczach:

\begin{itemize}[noitemsep]
 \item \emph{comment} --- treść komentarza wpisana przez użytkownika,
 \item \emph{thread} --- obiekt JSON zawierający dane dotyczące aktywnego wątku (tego, w którym odpowiada użytkownik), które zawiera nastepujące informacje:
 \begin{itemize}[noitemsep]
  \item \emph{aaa} --- dsadsadasda
  \item \emph{dsadasdsa} --- dasczxczxcxz
 \end{itemize}
\end{itemize}

\begin{figure}[htb!]
\label{newcommentrequest}
  \begin{verbatim}
{
    "comment": "Treść nowego komentarza", 
    "thread": {
        "numberofcomments": 7, 
        "name": "Testowy tytul", 
        "timestamp": "1438513896000", 
        "lastcomment": "1439395572000", 
        "path": "/", 
        "fullpath": "/home/user/aaa/.Comments/1438513896000@#&$Testowy tytul", 
        "type": "thread", 
        "unreadcomment": false
    }
}
  \end{verbatim}
  \caption{Przykładowe żądanie wysyłane do serwera podczas odpowiedzi na wątek.}
\end{figure}

Po otrzymaniu od klienta żądania z tymi danymi serwer za pomocą ścieżki \texttt{fullpath} wewnątrz słownika \texttt{thread} pobierze z serwera NTP znacznik czasowy (aby uwiarygodnić przykład, założono, że pobrany znacznik czasowy to \texttt{1439396297000}) i umieści w podanej lokalizacji (\texttt{/home/\-user/\-aaa/\-.Comments/\-1438513896000\@\#\&\$Testowy tytul}) nowy plik z komentarzem o nazwie \texttt{1439396\-297000\@\#\&\$2HI7KRUNS\-SONIUJKM\-RWXGOTIZ\-SHBGFIH}:

\begin{figure}[htb]
\begin{verbatim}
           {
               "comment": "Treść nowego komentarza",
               "timestamp": "1439396297000",
               "history": [
                   {
                       "comment": "Treść nowego komentarza",
                       "timestamp": "1439396297000"
                   }
               ], 
               "uid": "2HI7KRUNSSONIUJKMRWXGOTIZSHBGFIH",
               "readby": {
                   "2HI7KRUNSSONIUJKMRWXGOTIZSHBGFIH": "1439395604000"
               }
           }
\end{verbatim}
  \caption{Plik komentarza zapisany w wyniku otrzymania przez serwer  wyżej zaprezentowanego żądania.}
\end{figure}

\subsubsection*{Edycja komentarzy}

\begin{figure}[h!]
  \vspace{5pt}
  \begin{center}
    \includegraphics[width=260pt]{figures/meteditcomment.png}
  \end{center}
  \caption{Komunikacja w systemie \emph{TeamSync} podczas wywoływania metody \texttt{editcomment} --- edycja komentarzy.}
  \label{picmetdeletefolder}
\end{figure}

Modyfikacja komentarza może być dokonana tylko przez jego autora. Dzięki zastosowaniu takiego podejścia obniżone zostało ryzyko wystąpienia konfliktu podczas synchronizacji, ponieważ liczba osób mogących edytować plik komentarza została zredukowana do jednej.

Po zatwierdzeniu zmian do serwera wysyłane są dane niezbędne do odnalezienia komentarza w systemie plików i dopisania następnego wpisu do słownika \texttt{history} znajdującego się wewnątrz pliku z wypowiedzią wypowiedzi. Informacje wysyłane są za pomocą metody \texttt{POST} na adres \texttt{http://<adres oraz port\- serwera TeamSync>/\-editcomment}. Tymi danymi są:

\begin{itemize}[noitemsep]
  \item lokalizacja wątku --- pełna ścieżka folderu przechowującego pliki komentarzy oraz plik z metadanymi w dyskusji (plik \texttt{meta}),
  
  \item cały obiekt JSON zawierający dane komentarza w formie, w jakiej jest on odczytywany z pliku, z tą różnicą, że w zmiennej \texttt{comment} (zawierającej treść komentarza) znajduje się nowa, zmodyfikowana przez użytkownika wypowiedź.
\end{itemize}

Po otrzymaniu żądania zawierającego powyższe informacje serwer odtwarza tytuł pliku komentarza, posługując się zmiennymi \texttt{timestamp} oraz \texttt{uid}, otrzymując pełną ścieżkę komentarza. Wczytuje plik komentarza zapisany na dysku, modyfikuje zmienną \texttt{comment} treścią podaną przez użytkownika i dodaje do słownika \texttt{history} nowy element zawierający pobrany znacznik czasowy z serwera \emph{NTP} oraz nową treść wypowiedzi. Następnie zapisuje komentarz w tym samym pliku.

Aby zaoszczędzić czas, który tracony jest na odczytywanie pliku z systemu plików, serwer mógłby pominąć ten etap i --- po dodaniu wpisu do zmiennej \texttt{history} --- od razu komentarz zapisać. Jednak dane przesyłane z przeglądarki do serwera zawierają informacje nadmiarowe dotyczące interfejsu graficznego, które są zbędne dla serwera. Do poprawnego działania funkcji serwerowi wystarczyłyby dane:

\begin{itemize}[noitemsep]
  \item \emph{comment} --- nowa treść komentarza,
  
  \item \emph{timestamp} --- znacznik czasowy utworzenia komentarza (potrzebne do identyfikacji pliku w folderze wątku),
  
  \item \emph{uid} --- autor komentarza (potrzebne do identyfikacji pliku w folderze wątku),
  
  \item \emph{fullthreadpath} --- pełna ścieżka wątku, w którym komentarz jest fizycznie zapisany.
\end{itemize}

Jednakże zdecydowano się na przesłanie całej struktury komentarza ze względu na niewielką nadmiarowość danych oraz większe możliwości manipulacji zapisywaniem komentarza podczas ewentualnych ulepszeń aplikacji TeamSync w przyszłości.

\subsection{Wątki}

Użytkownicy mają możliwość tworzyć komentarze oraz je edytować, natomiast \emph{TeamSync} nie umożliwia usuwania ich ze względu na zachowanie spójności logicznej dyskusji. W przypadku wątków możliwości użytkowników zostały dodatkowo ograniczone --- edytowanie wątku zostało zminimalizowane wyłącznie do edycji treści pierwszego komentarza. Ze względu na fakt, iż była ona omówiona wcześniej (sekcja \ref{comments}), omówiona zostanie wyłącznie procedura dodawania do systemu nowej dyskusji.

\subsubsection*{Tworzenie nowego wątku}

\begin{figure}[h!]
  \vspace{5pt}
  \begin{center}
    \includegraphics[width=250pt]{figures/metwritenewthread.png}
  \end{center}
  \caption{Komunikacja w systemie \emph{TeamSync} podczas wywoływania metody \texttt{writenewthread} --- tworzenie nowego watku.}
  \label{picmetdeletefolder}
\end{figure}

Użytkownik tworząc nowy wątek, wysyła --- poprzez aplikację kliencką w przeglądarce --- do serwera żądanie \texttt{HTTP} za pomocą metody \texttt{POST} na adres \texttt{http://<adres oraz port\- serwera TeamSync>/\-writenewthread}. W zapytaniu przesyłane sa następujące parametry:

\begin{itemize}[noitemsep]
  \item \emph{topic} --- tytuł wątku wprowadzony przez użytkownika,
  
  \item \emph{comment} --- treść pierwszego komentarza wprowadzona przez użytkownika,
  
  \item \emph{folderpath} --- bezwzględna ścieżka synchronizowanego folderu w lokalnym systemie operacyjnym użytkownika,
  
  \item \emph{insidepath} --- lokalizacja wątku wewnątrz folderu nie uwzględniając samego katalogu przechowującego wątek,
  
  \item \emph{fileabout} --- nazwa pliku wybranego przez użytkownika, którego dotyczyć będzie wątek.
\end{itemize}

Po otrzymaniu żądania serwer w pierwszej kolejności pobiera znacznik czasowy z serwera NTP, który jest potrzebny do wygenerowania nazwy folderu z nowym wątkiem łączącej ze sobą pobrany znacznik, identyfikator użytkownika oraz separator nazwy (dokładny opis separatora znajduje się w sekcji \ref{filenames}). Przyjmując jako separator użyty w implementacji ciąg znaków: \texttt{@\#\&\$}, nazwa katalogu wygląda następująco:

\begin{verbatim}
               < znacznik czasowy >@#&$< identyfikator użytkownika >
\end{verbatim}

Następnie serwer ustala pełną ścieżkę wątku z przekazanych w żądaniu parametrów \texttt{folderpath} oraz \texttt{insidepath}:

\begin{verbatim}
            folderpath/.Comments/insidepath/< nazwa katalogu z wątkiem >
\end{verbatim}

Po ustaleniu pełnej lokalizacji wątku serwer tworzy folder o podanej wyżej ścieżce i umieszcza w nim plik \texttt{meta} uzupełniając go wartościami podanymi przez użytkownika w żądaniu. Poniżej przedstawiono przykładowe parametry przesyłane w celu stworzenia nowego wątku w postaci obiektu \texttt{JSON}.

\begin{figure}[htb]
\begin{verbatim}
                {
                    "topic": "Testowy tytuł", 
                    "comment": "Przykładowa treść komentarza", 
                    "insidepath": "/abc", 
                    "fileabout": "aaa", 
                    "folderpath": "/home/user/testowy katalog"
                }
\end{verbatim}
  \caption{Parametry przesyłane w przykładowym zapytaniu tworzącym nowy wątek.}
\end{figure}

Według parametrów zawartych w przykładzie nowy wątek będzie miał tytuł \texttt{Testowy tytuł}, a treść pierwszego komentarza to \texttt{Przykładowa treść komentarza}. Lokalizacja uwspólnionego folderu, wewnątrz którego zamieszczany jest wątek, to \texttt{/home/user/\-testowy katalog}, natomiast wątek umieszczony jest jeden poziom głębiej, w katalogu \texttt{abc}. Dodatkowo utworzona dyskusja dotyczy pliku o pełnej ścieżce \texttt{/home/user/\-testowy katalog/\-abc/aaa}.

W wyniku otrzymania powyższego żądania --- zakładając, że pobrany znacznik czasowy to \texttt{1441566\-810000}, a identyfikator użytkownika zakładającego wątek to \texttt{FWZPQQKU\-E3JPQKWY5\-57FERJ4S\-D3BSFMM} --- serwer utworzy wątek w lokalizacji o pełnej ścieżce:

\begin{verbatim}
       /home/user/testowy katalog/.Comments/abc/
                         1441566810000@#&$FWZPQQKUE3JPQKWY557FERJ4SD3BSFMM
\end{verbatim}

Wewnątrz tego folderu zostanie umieszczony plik \texttt{meta} oraz plik pierwszego komentarza o podanych poniżej zawartościach.

\begin{figure}[htb]
\begin{verbatim}
                {
                    "topic": "Testowy tytuł", 
                    "timestamp": "1441566810000", 
                    "fileabout": "/abc/aaa", 
                    "uid": "FWZPQQKUE3JPQKWY557FERJ4SD3BSFMM"
                }
\end{verbatim}
  \caption{Zawartość pliku \texttt{meta} utworzonego w wyniku otrzymania przykładowego żądania.}
\end{figure}

\begin{figure}[htb]
\begin{verbatim}
           {
               "comment": "Przykładowa treść komentarza", 
               "timestamp": "1441566810000", 
               "history": [
                   {
                       "comment": "Przykładowa treść komentarza", 
                       "timestamp": "1441566810000"
                   }
               ], 
               "readby": {
                   "FWZPQQKUE3JPQKWY557FERJ4SD3BSFMM": "1441566810000"
               }, 
               "uid": "FWZPQQKUE3JPQKWY557FERJ4SD3BSFMM"
           }
\end{verbatim}
  \caption{Zawartość pliku z pierwszym komentarzem utworzonego w wyniku otrzymania przykładowego żądania.}
\end{figure}

W pliku \texttt{meta} zapisanym w folderze w wyniku otrzymania przez serwer przykładowego żądania umieszczony został tytuł \texttt{Testowy tytuł} wpisany przez użytkownika, znacznik czasowy zwrócony przez serwer NTP, identyfikator autora wątku oraz wewnętrzna ścieżka do pliku, na którego temat powstała dyskusja. Jeśli wątek nie dotyczyłby żadnego z plików, wartość \texttt{fileabout} byłaby równa \texttt{<brak>}.

Zapisywanie plików komentarzy zostało dokładniej opisane w sekcji \ref{comments}.

\section{Aplikacja przeglądarkowa}

Wśród podstawowych założeń implementacyjnych aplikacji \emph{TeamSync} --- poza założeniami wynikającymi z zastosowania systemu \emph{BitTorrent Sync} jako narzędzia odpowiedzialnego za wymianę danych --- znajdują się wymagania dotyczące interakcji człowieka z systemem. Podstawowymi założeniami --- oprócz prostoty interfejsu oraz łatwości jego obsługi --- w tym zakresie są: częstość odświeżania informacji oraz szybkość działania.

Aby sprostać powyższym wymaganiom i zachować interfejs w statycznym środowisku --- jakim jest serwer WWW --- zastosowano w pracy asynchroniczne zapytania (\emph{AJAX} \cite{ajax}) między częścią serwerową i kliencką. Podczas niemal każdej z czynności wykonywanych przez użytkownika w interfejsie odświeżana jest większa część danych pokazywanych użytkownikowi:

\begin{itemize}[noitemsep]
  \item komentarze,
  \item listy wątków,
  \item listy plików/katalogów przechowywanych w synchronizowanym folderze.
\end{itemize}

\subsubsection*{Two-way data binding}

Ważnym mechanizmem umożliwiającym dynamiczną zmianę treści strony unikając jej całkowitego przeładowania, jest tzw. ,,\emph{two-way data binding}'' zaimplementowany przez twórców narzędzia \emph{AngularJS}. Interfejs działający w tym systemie wyświetla elementy \emph{modelu}, które użytkownik zaimplementuje --- w przypadku aplikacji \emph{TeamSync} wewnątrz modelu znajdują się np. wszystkie listy przechowujące dane (listy plików, wątków, komentarzy, identyfikator użytkownika, itp.) albo ustawienia aplikacji.

Ogólna koncepcja metody ''\emph{two-way data binding}'' polega na zrealizowaniu jednocześnie dwóch następujących założeń dotyczących sposobu prezentacji w interfejsie danych pochodzących z modelu:

\begin{enumerate}[noitemsep]
  \item Jeśli dane zmienią się wewnątrz modelu, zmiana nastąpi również w interfejsie graficznym.
  
  \item Jeśli dane prezentowane w interfejsie graficznym zostaną zmienione, zmiana nastąpi również w \emph{modelu}.
\end{enumerate}

W ten sposób --- korzystając z pierwszego założenia --- jeśli wewnątrz kodu przeglądarki nastąpi odświeżenie dowolnej zmiennej (np. listy plików) poprzez zapytanie asynchroniczne, zostanie ona z natychmiastowym skutkiem zmieniona w interfejsie graficznym. Przykładem wykorzystania drugiego założenia może być sytuacja, w której użytkownik chce posortować wątki według żądanego kryterium. Wprowadzając odmienne kryterium niż obecne, wywołuje szereg operacji nie na zmiennych wyświetlanych w interfejsie, lecz na zmiennych wewnątrz modelu.

Dokładniejszy opis i przykłady obydwóch powyższych założeń zostaną przedstawione w następnych podrozdziałach.

\subsection{Wyświetlanie plików}

Podczas odświeżania listy plików, w kodzie \emph{javascript} interfejsu wysyłane jest asynchroniczne zapytanie metodą \texttt{POST}, a cała struktura, którą przeglądarka otrzyma w odpowiedzi od serwera, umieszczana jest wewnątrz zmiennej \texttt{files} reprezentującej listę plików. Aby przedstawić całą otrzymaną listę w interfejsie użytkownika, należy umieścić ją w strukturze, która umożliwi czytelne odwzorowanie listy. W tym celu posłużono się znacznikami \texttt{ul} oraz \texttt{li}, które są stworzone z myślą o prezentacji list (ul = unordered list). Przykładowe użycie znaczników \texttt{ul} oraz \texttt{li} wygląda następująco:

\begin{figure}[htb]
\begin{verbatim}
                            <ul>
                                <li>Element 1</li>
                                <li>Element 2</li>
                                <li>Element 3</li>
                            </ul>
\end{verbatim}
\end{figure}

\emph{AngularJS} umożliwia dynamiczną manipulację elementami listy poprzez dyrektywę \texttt{ng-repeat} dodawaną do tego znacznika, który ma zostać powtórzony. W przypadku powyższego przykładowego kodu dyrektywa \texttt{ng-repeat} zostanie dodana do pierwszego znacznika \texttt{li}. Jeśli w modelu istniałaby zmienna \texttt{list}, będąca tablicą zawierającą trzy elementy o typie znakowym: ,,\texttt{Element 1}'', ,,\texttt{Element 2}'' oraz ,,\texttt{Element 3}'', poniższy kod w pliku \emph{HTML} byłby jednakowy z wcześniejszym:

\begin{figure}[htb]
\begin{verbatim}
                 <ul>
                     <li ng-repeat="item in list">{{ item }}</li>
                 </ul>
\end{verbatim}
\end{figure}

Dyrektywa \texttt{ng-repeat} tworzy pętlę przechodzącą po wszystkich elementach listy, która wewnątrz każdego znacznika \texttt{li} umieszcza element, którego aktualnie dotyczy iteracja. Podwójny nawias ,,\{'' służy do umieszczania zawartości przechowywanych wewnątrz zmiennych, które służą jako iterator (w powyższym przypadku iteratorem jest zmienna \texttt{item}).Jeśli modyfikacji (na przykład podczas asynchronicznego zapytania do serwera i zaktualizowania listy) ulegnie lista \texttt{list}, wewnątrz której odbywa się iteracja w dyrektywie \texttt{ng-repeat}, zmiana zostanie przeniesiona na interfejs. Spełnione zostanie pierwsze założenie metody \emph{two-way data binding} --- zmiana modelu wpłynie na interfejs graficzny.

Dyrektywa \texttt{ng-repeat} została wykorzystana w aplikacji \emph{TeamSync} np. do prezentowania listy plików. Zmienna \texttt{files}, w której umieszczane są elementy pochodzące z odpowiedzi serwera, stanowi listę, wewnątrz której dyrektywa \texttt{ng-repeat} iteruje, wyświetlając odpowiednie informacje użytkownikowi.

\begin{figure}[htb]
\begin{verbatim}
           <ul>
               <li ng-repeat="file in files">{{ file.name }}</li>
           </ul>
\end{verbatim}
\end{figure}

Ze względu na założenia metody \emph{two-way data binding}, w momencie odebrania odpowiedzi od serwera, przeglądarka natychmiast wyświetli aktualną listę plików, czego efektem jest niezbędna dynamika aplikacji. Ponieważ zawarte wewnątrz listy \texttt{files} elementy są obiektami \texttt{JSON}, wewnątrz podwójnych nawiasów klamrowych musiała zostać umieszczona któraś z wartości obiektu, a nie cały obiekt. Skoro użytkownik musi widzieć nazwę pliku, aby móc nawigować po synchronizowanym folderze, została użyta wartość obiektu \texttt{file}, której kluczem jest \texttt{name} (nazwa).

W podobny sposób działają wszystkie dynamiczne elementy strony interfejsu graficznego, które są modyfikowane zarówno przez serwer, jak i użytkownika. Przykładem zmiennych, które prezentowane sa w podobny sposób do plików, są:

\begin{itemize}[noitemsep]
  \item lista folderów,
  \item lista wszystkich użytkowników w folderze,
  \item lista komentarzy,
  \item lista wątków.
\end{itemize}


\subsection{Filtrowanie komentarzy}

Wyszukiwanie wewnątrz komentarzy wpisywanych przez użytkownika fraz w aplikacji \emph{TeamSync} odbywa się w całości po stronie klienta, dzięki czemu jest bardzo szybkie i dynamiczne. Filtr jest uaktualniany i zaczyna działać podczas wprowadzenia każdego znaku. Odbywa się to dzięki użyciu słowa kluczowego \texttt{filter} dostępnego w systemie \emph{AngularJS}, którego umieszczenie w odpowiedni sposób wewnątrz kodu HTML aplikacji spowoduje wykluczenie ze zbioru danych elementów niepasujących do wzorca.

Użycie funkcji \texttt{filter} zostanie opisane na kodzie HTML z wcześniejszego przykładu, do którego dodano funkcjonalność filtrowania:

\begin{figure}[htb]
\begin{verbatim}
      <ul>
          <li ng-repeat="item in list | filter:testFilter">{{ item }}</li>
      </ul>
\end{verbatim}
\end{figure}

Znak ,,\texttt{|}'' jest znakiem, który powoduje poddanie zbioru danych sprzed tego znaku (w tym przykładzie jest to lista \texttt{list}) działaniu operacji po tym znaku --- filtrowaniu, lub sortowaniu (w tym przykładzie jest to filtrowanie). Słowo \texttt{filter} jest słowem kluczowym oznaczającym wykonywanie filtrowania na zbiorze danych \texttt{list}. \texttt{testFilter} jest zmienną pochodzącą z modelu, wewnątrz której przechowywana jest wartość filtru. Jeśli byłby spełnione poniższe warunki:

\begin{figure}[htb]
\begin{verbatim}
               testFilter = "1"
               list = ["Element 1", "Element 2", "Element 3"]
\end{verbatim}
\end{figure}

Wewnątrz znacznika \texttt{ul} zostałyby wyświetlone wyłącznie te elementy \texttt{li}, których zawartość zawierałaby znak ,,$1$''. Posługując się obecnym przykładem, użytkownik wyświetlając powyższy kod z powyższymi danymi wewnątrz modelu zobaczyłby w swojej przeglądarce wyłącznie pierwszy element o treści \texttt{Element 1}.

Wewnątrz aplikacji \emph{TeamSync} funkcjonalność słowa kluczowego \texttt{filter} została wykorzystana do filtrowania tresci komentarzy w poszukiwaniu żądanego przez użytkownika słowa. Wszystkie komentarze aktualnie wyświetlane na ekranie zostają poddawane filtracji w poszukiwaniu wprowadzonej frazy. W uproszczonej wersji kod HTML tej funkcjonalności w aplikacji \emph{TeamSync} wygląda w następujący sposób:

\begin{figure}[htb]
\begin{verbatim}
         <ul>
             <li ng-repeat="comment in comments | filter: searchphrase">
         </ul>
\end{verbatim}
\end{figure}

Zmienna \texttt{comments} jest listą, do której asynchronicznie pobierane są z serwera komentarze w formie obiektów \texttt{JSON}. Podobnie jak w wyświetlaniu plików, również tutaj podczas aktualizacji tej zmiennej w modelu (dzięki technice \emph{two-way data binding}) natychmiast aktualizowane są elementy w interfejsie graficznym. Po słowie kluczowym \texttt{filter} --- sygnalizującym działanie filtra zbioru danych \texttt{comments} --- znajduje się zmienna \texttt{searchphrase}, której wartość jest ciągiem znaków wyszukiwanym wśród komentarzy. Wyświetlany zbiór wypowiedzi (spośród listy \texttt{comments}) zostanie ograniczony tylko do tych, które zawierają frazę wpisaną do zmiennej \texttt{searchphrase}.

Podczas tej operacji system \emph{AngularJS} skorzysta z drugiego założenia metody \emph{two-way data binding} --- użytkownik wprowadzając w interfejsie graficznym szukaną frazę, automatycznie uaktualni zmienną \texttt{searchphrase} w modelu. Z kolei zmienna \texttt{searchphrase} pobrana z modelu zostanie użyta do wyświetlenia zubożonego zbioru komentarzy o te, które nie pasują do wpisanego wzorca.

W systemie \emph{AngularJS} możliwe jest stosowanie wielu filtrów na tym samym zbiorze danych. W aplikacji \emph{TeamSync} zostało to wykorzystane, aby dodatkowo móc wyfiltrować te wiadomości, które zostały umieszczone przez konkretnych użytkowników.
\input{chapters/07-zakonczenie.tex}

% All appendices and extra material, if you have any.
\cleardoublepage\appendix%
\input{chapters/0a-instrukcja.tex}
% \input{chapters/0b-pisanie-w-latexu.tex}

% Bibliography (books, articles) starts here.
\bibliographystyle{plplain}{\raggedright\sloppy\small\bibliography{chapters/bibliography}}

% Colophon is a place where you should let others know about copyrights etc.
\ppcolophon

\end{document}
